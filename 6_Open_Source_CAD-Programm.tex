
Das Open Source ECAD-Programm KiCAD ist eine Anwendung zum Erstellen von Schaltplänen und elektronischen Leiterplatten. 

\subsection{Schaltplan}
Beim Schaltplanentwurf gilt es auf gewisse Regeln zu achten, zu dem sollte die Übersichtlichkeit des Schaltplans nicht außer Acht gelassen werden.
So sollten beispielsweise bei Anschluss längerer Leitungen, Kondensatoren zum Ausfiltern eingefangener Funksignale und anderer EMV-Belastungen, angebracht werden. Die Bauteile sollten möglichst so geführt und platziert werden, wie sie später im Layout liegen sollten, wie z.B. gemeinsame Potenzialbezugspunkte.\\
Im Schaltplan wurden die Baugruppen Empfänger, Sender, Hochsetzsteller und die Anschlüsse vom Controller getrennt und so positioniert, dass die beste Übersicht dargestellt wird.\\
Die wichtigen Grenzpunkte wurden mit einem null Ohm Widerstand bestückt, um etappenweise die Platine in Betrieb nehmen zu können somit ist es bei der Erstinitialisierung
sicher zustellen das eine Baugruppe eine andere Beschädigen kann. \\   

\subsection{Platinenlayout}
Beim Entwerfen eines Platinenlayouts gibt es viele Möglichkeiten ein Ergebniss zu erzielen. So können alle Bauteile so angeordnet werden, dass alle parallelen Bauteile nebeneinander aufgereiht werden und die in Reihe dazu liegenden Bauteile darunter angeordnet sind. So sähe die Platine zwar ähnlich eines Kontaktplanes aus, allerdings ist diese Variante aus Sicht der EMV nicht sonderlich empfehlenswert.\\
Eine andere Möglichkeit wäre, die Bauteile im Schaltplan in Gruppen zusammenzulegen und den Schaltplan auf der Platine exakt nachzubilden. Auch bei dieser Variante ergeben sich gelegentlich Probleme, was die Führung der Leitungen und vor allem den Verlauf der Ströme angeht.\\
So sollte ein Augenmerk auf den stromführenden Leitungen liegen. Je höher der Strom ist, desto breiter und kürzer ist die Leitung auszulegen, um weniger EMV-Störungen zu erzeugen. Auch sollte die Rückführung (GND) günstigerweise als eigene Leiterschicht ausgeführt werden, um einen großen Leiterquerschnitt zu gewährleisten. So kann bei der Rückführung der Ströme auch das Risiko vermieden werden, durch Bildung von größeren Schleifen, Antennen zu erzeugen. Die GND-Schicht sollte so wenig wie möglich unterbrochen werden, vorallem sind Unterbrechungen quer zur Stromflussrichtung zu vermeiden. Zusätzlich ist zu beachten, dass Kondensatoren, zur Verringerung von Störungen(EMV), nahe digitalen Bauteilen angebracht werden sollten. Die optimale Platzierung ist direkt am VDD oder VCC des ICs, so dass die Leiterbahn vom IC mit einem höheren Querschnitt am Kondensator liegt und dann mit leicht verringertem Querschnitt weiterverläuft.


