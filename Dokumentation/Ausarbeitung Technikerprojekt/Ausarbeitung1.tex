%Einfache Vorlage für eine mit Latex realisierte Hausarbeit von http://www.studieren-info.de
%Du kannst diese Vorlage für deine Hausarbeit beliebig anpassen%


%-------------------
%Beginn des Kopfbereiches
%-------------------

%Wir verwenden eine DIN-A4-Seite und die Schriftgröße 12.
\documentclass [11pt,a4paper,bibliography=totoc]{scrreprt}%Einfaches Dokument{article, scrartcl} komplexere Dokumente/Studienarbeiten {report, scrreprt} Bücher {book, scrbook} Präsentationen {beamer, seminar, texpower} Briefe {g-brief, scrlttr2}

%Diese drei Pakete benötigen wir für die Umlaute, Deutsche Silbentrennung etc.
%Apple-Nutzer sollten anstelle von \usepackage[latin1]{inputenc} das Paket \usepackage[applemac]{inputenc} verwenden
\usepackage[utf8]{inputenc}%[utf8] Umlaute direkt eingabe [latin1] für unixoide und windows
\usepackage[ngerman]{babel}
\usepackage[T1]{fontenc}
\usepackage{tikz}
\usetikzlibrary{shapes,arrows}
%\usepackage{scrhack}
\usepackage[margin=9pt,font=small,labelfont=bf,textfont=it,belowskip=.3cm]{caption}
%Verwendung von vielen Farben
%\usepackage[usenames,dvipsnames]{color}
\usepackage{color}
% deutsches Literaturverzeichnis
\usepackage{bibgerm}
\usepackage{listings}%um Quellcode vernünftig einpflegen zu können
\lstset{language=C} 
\definecolor{codegreen}{rgb}{0,0.6,0}
\definecolor{codegray}{rgb}{0.5,0.5,0.5}
\definecolor{codepurple}{rgb}{0.58,0,0.82}
\definecolor{backcolour}{rgb}{0.95,0.95,0.92}
 
\lstdefinestyle{mystyle}{
    backgroundcolor=\color{backcolour},   
    commentstyle=\color{codegreen},
    keywordstyle=\color{magenta},
    numberstyle=\tiny\color{codegray},
    stringstyle=\color{codepurple},
    basicstyle=\footnotesize,
    breakatwhitespace=false,         
    breaklines=true,                 
    captionpos=b,                    
    keepspaces=true,                 
    numbers=left,                    
    numbersep=5pt,                  
    showspaces=false,                
    showstringspaces=false,
    showtabs=false,                  
    tabsize=2
}
 
\lstset{style=mystyle}


%nutzung der verlinkten formeln
\usepackage{amsmath}
%Das Paket erzeugt ein anklickbares Verzeichnis in der PDF-Datei.
\usepackage{hyperref}
%Das Paket wird für die anderthalb-zeiligen Zeilenabstand benötigt
\usepackage{setspace}

%erweitrte mathematische Tabellenformatierung
\usepackage{array}
% Grafiken verwenden
\usepackage{graphicx}
\usepackage{float}
\usepackage{struktex}
\usepackage{eurosym}
\usepackage{tabularx}
\usepackage{textcomp}
%Funktion zum skallieren
\makeatletter
\def\ScaleIfNeeded{%
\ifdim\Gin@nat@width>\linewidth
\linewidth
\else
\Gin@nat@width
\fi
}
\makeatother

%\tiny{winzig}
%\small{klein}
%\large{groß}
%\Large{bisschen größer}
%\huge{riesig}
%\Huge{Riesig}

%\textbf{Fett}
%\textit{Kursiv}
%\emph{Kursiv}	
%\textsl{schief}
%\textsc{Kapit\"alchen}
%\textsf{Sans Serif}
%\textrm{Roman}
%\texttt{Schreibmaschine}
%\textnormal{Normale Schrift}
%\underline{unterstrichen}
%\footnote{XXX}
%\includegraphics[width=80pt]{Unterschrift.jpg}

%\textcolor{white}{bla...}
%{GreenYellow, Yellow, Goldenrod, Dandelion, Apricot, Peach, Melon, YellowOrange, Orange, BurntOrange, Bittersweet, RedOrange, Mahogany, Maroon, BrickRed, Red, OrangeRed, RubineRed, WildStrawberry, Salmon, CarnationPink, Magenta, VioletRed, Rhodamine, Mulberry, RedViolet, Fuchsia, Lavender, Thistle, Orchid, DarkOrchid, Purple, Plum, Violet, RoyalPurple, BlueViolet, Periwinkle, CadetBlue, CornflowerBlue, MidnightBlue, NavyBlue, RoyalBlue, blue, Blue, Cerulean, Cyan, ProcessBlue, SkyBlue, Turquoise, TealBlue, Aquamarine, BlueGreen, Emerald, JungleGreen, SeaGreen, Green, ForestGreen, PineGreen, LimeGreen, YellowGreen, SpringGreen, OliveGreen, RawSienna, Sepia, Brown, Tan, Gray}

%Einrückung eines neuen Absatzes
\setlength{\parindent}{0em}
%Definition der Ränder
\usepackage[paper=a4paper,left=2.5cm, right=2.0cm, top=2.5cm, bottom=2.0cm]{geometry}
%anpassung der Farben der Überschriften
\addtokomafont{disposition}{\color{black}}
%Abstand der Fußnoten
\deffootnote{1em}{1em}{\textsuperscript{\thefootnotemark\ }}

\addtolength{\footskip}{-1cm}% Fußbereich 1 cm höher setzen
%section size 
%\usepackage{titlesec}
%\titleformat{\chapter}[display]
%{\normalfont%
%    \huge% %change this size to your needs for the first line
%    \bfseries}{\chaptertitlename\ \thechapter}{15pt}{%
%    \Huge %change this size to your needs for the second line
%    }
%%\titleformat*{\chapter}{\fontsize{15}{20}\selectfont}
%\titleformat*{\section}{\fontsize{13}{20}\selectfont}
%\titleformat*{\subsection}{\fontsize{12}{17}\selectfont}

%Regeln, bis zu welcher Tiefe (section,subsection,subsubsection) überschriften angezeigt werden sollen (Anzeige der überschriften im Verzeichnis / Anzeige der Nummerierung)
\setcounter{tocdepth}{2}
\setcounter{secnumdepth}{2}

%-------------------
%Ende des Kopfbereiches
%-------------------



%-------------------
%Hier beginnt der Text deiner Hausarbeit
%-------------------
\begin{document}


%Beginn der Titelseite
\thispagestyle{empty}
\begin{center}
\begin{Huge}
\textcolor{blue}{\textbf{Entwicklung und testen eines Ultraschall-Entfernungsmessers als Vorbereitung eines Produktentwurfes}}
\end{Huge}
\rule{\textwidth}{.4pt}
\vspace{1.5cm}
% Weiter in großer Schrift

\huge{\textbf{Projektarbeit}}\\
\begin{Large}
erstellt an der\\
Fachschule für Technik des Carl-Severing-Berufskolleg\\
für Metall- und Elektrotechnik der Stadt Bielefeld\\
\includegraphics[width=100pt]{Abbildungen/CSBlogo.png}\\

Erstellt durch:\\
\vspace{12pt}
Eduard Meiser\\Omar Hachimi \\Stephan Dannat\\FET6A\\
\vspace{12pt}
in Zusammenarbeit mit der Fa. Tinkerforge\\
betreut durch\\
Herr Simon\\
Bielefeld, \today
\end{Large}
\end{center}
\newpage
\thispagestyle{empty}
\textbf{\Large{Persönliche Erklärung}}\vspace{10pt}

Hiermit bestätigen wir, dass die vorliegende Arbeit selbstständig verfasst und keine anderen als die angegebenen Hilfsmittel benutzt wurden. Die Stellen der Arbeit, die dem Wortlaut oder dem Sinn nach anderen Werken (dazu zählen auch Internet-quellen) entnommen sind, wurden unter Angabe der Quellen kenntlich gemacht.

\vspace{50pt}


\noindent Bielefeld,\noindent\rule{4cm}{.4pt} \hfill\rule{5cm}{.4pt}\par
\hfill Eduard Meiser 

\vspace{30pt}
%\noindent\rule{5cm}{.4pt}
\hfill\rule{5cm}{.4pt}\par
%\noindent Ort, Datum
\hfill Omar Hachimi 

\vspace{30pt}
%\noindent\rule{5cm}{.4pt}
\hfill\rule{5cm}{.4pt}\par
%\noindent Ort, Datum 
\hfill Stephan Dannat 
%				Inhaltsverzeichnis
\setcounter{page}{0}
\newpage 
\pagenumbering{roman} 
\tableofcontents 
\newpage
%				Start der eigentlichen Arbeit
\newpage
\setcounter{page}{0}
\pagenumbering{arabic}

\chapter{Einleitung}
\section{Zusammenfassung}
Bei dieser Projektarbeit wurden Messungen an einem zuvor selbst entworfenen Prototyp eines Ultraschall-Entfernungsmessers durchgeführt. Das geschah im Auftrag der Firma Tinkerforge, um herauszufinden, wie sich ein solcher Sensor realisieren lässt. Dabei gab es zwei Möglichkeiten und entsprechend dieser auch zwei unterschiedliche Prototyp-Versionen. Begonnen wurde mit einer Variante mit getrenntem Sende- und Empfangsbetrieb, auf zwei Platinen aufgebaut. Danach wurde eine zweite Variante mit kombiniertem Betrieb, auf einer Platine aufgebaut. Diese zweite Prototyp-Version besaß nur noch eine Ultraschallkapsel für den Sende- und Empfangsbetrieb Trotz mehrerer notwendiger Veränderungen an den Platinen ließen sich Aussagekräftige Messungen durchführen.
\section{Lastenheft}
\subsection{Über Tinkerforge}
Die  Tinkerforge  GmbH  wurde  Ende  2011  mit  dem  Ziel  gegründet,  die  Handhabung eingebetteter  Systeme  zu  vereinfachen.  Das  Tinkerforge  Baukastensystem  besteht  aus aktuell fast 80 verschiedenen Modulen, die vom Anwender flexibel für die jeweilige Aufgabe  kombiniert  werden  können.  Zu  den  Modulen  zählen  diverse  Sensor-  Aktor-  und Schnittstellenmodule, die alle über Hochsprachen wie C\#, Python und Java gesteuert werden können. Tinkerforge unterstützt aktuell 17 verschiedene Programmiersprachen. Sowohl Hardware als auch die Software aller Module sind OpenSource. Die Stärke des Baukastensystems  ist  aus  Anwendersicht  die  enorme  Flexibilität,  die  Einfachheit  und
die Schnelligkeit mit der Projekte realisiert werden können. Es eignet sich daher besonders im Bereich Rapid Prototyping. Daher findet das Tinkerforge Baukastensystem Anwendung in vielen Forschungsinstituten, in diversen Entwicklungsabteilungen bekannter Automobilhersteller und Ingenieurbüros.
\subsection{Motivation}
Diese Technikerarbeit soll die Grundlage zur Entwicklung eines Entfernungssensors für das Baukastensystem bilden, der auf einer Ultraschall-Entfernungsmessung basiert. Das Baukastensystem verfügt aktuell über so einen Sensor. Bei diesem handelt es sich aber im wesentlichen um ein zugekauftes Modul, welches nicht die gewünschten Leistungen liefert. Daher soll an einem zu entwerfenden Prototypen Forschung betrieben werden, um eine eigene Lösung entwerfen zu können.
\subsection{Aufgabenbeschreibung}
Innerhalb dieser Arbeit soll der Entwurf eines Prototypen des Entfernungssensors und die damit verbundene Forschung durchgeführt werden. Dabei ist durch Recherche zu erarbeiten, welche Möglichkeiten zur Realisierung zur Verfügung stehen. Durch Messungen am Prototypen soll festgestellt werden, welche dieser Möglichkeiten funktional und finanziell realisierbar sind, um ein eigenes Produkt zu erstellen. Sollte im Anschluss noch die Möglichkeit bestehen, sind die Ergebnisse in ein serienreifes Modul umzusetzen.
\newline\newline
Diverse Teilaufgaben sind zu erledigen: \newline
\begin{itemize}
\item \textbf{Recherche}\newline
Zu  Anfang  muss  recherchiert  werden,  welche  Möglichkeiten  es  gibt  mittels  Ultraschall eine Entfernung zu ermitteln und wie diese technisch umgesetzt werden können. Zusätzlich müssen die Techniker sich mit dem Tinkerforge Baukastensystem und seiner internen Funktionsweise vertraut machen.
\item \textbf{Bauteilauswahl}\newline
Abhängig  von  der  gewählten  technischen  Umsetzung  müssen  geeignete  Komponenten ausgewählt werden. Die Auswahl sollte auch unter dem Gesichtspunkten Preis, der Bauteilverfügbarkeit und der technischen Anforderungen erfolgen.\\
\item \textbf{Schaltplanentwurf und Layouterstellung}\newline
Von  Tinkerforge  wird  das  Open  Source  CAD  Programm  KiCad  verwendet.  Mit diesem Programm ist ein Schaltplan für den Prototypen und anschließend ein Leiterplattenlayout zu erstellen.
\item \textbf{Leiterplattenbestückung}\\
Die erstellte Leiterplatte wird von Tinkerforge in Auftrag gegeben. Diese muss mit den gewählten Komponenten bestückt werden. Die Tinkerforge GmbH stellt dazu die notwendigen Werkzeuge bereit.
\item \textbf{Einrichten und Einarbeitung in die Tinkerforge Toolchain}\\
Viele  Softwarekomponenten  werden  von  der  Tinkerforge  Toolchain  automatisch generiert. Um diese Nutzen zu können muss ein Linux System %in einer virtuellen Maschine
eingerichtet werden. Anschließend muss sich mit der Funktionsweise des Generators und der Softwareversionsverwaltung"Git" vertraut gemacht werden.
\item \textbf{Testsoftware und Forschung}\\
Um Messungen an der Hardware durchführen zu können gilt es Programmblöcke zu entwerfen, mit denen die einzelnen Funktionen der Baugruppen getestet werden können. So soll ermittelt werden, wie zum einen das Ultraschallsignal effektiv ausgegeben werden kann und wie sich die Signalamplitude auf die Reichweite auswirkt. Zum anderen gilt es zu recherchieren, wie das zurückkommende Signal verarbeitet werden kann. Auch soll erarbeitet werden, wie gut das Signal unter verschiedenen Bedingungen verarbeitet werden kann und ob eine zuverlässige Verarbeitungsqualität ohne großen Aufwand realisierbar ist.
\end{itemize}
\section{Management des Projektes}

\subsection{Trello}
Zur zeitlichen Planung und Übersicht des Ablaufes wurde auf das Onlinetool Trello zurückgegriffen. Dieses ist ein kostenfreies, webbasiertes Projektmanagementtool. Es ermöglicht den Gruppenmitgliedern gleichzeitig von verschiedenen Orten auf die Oberfläche zuzugreifen und Änderungen vorzunehmen. So kann ein Teilnehmer auch neue Termine mit Kennzeichnung der Fälligkeit für andere Gruppenmitglieder einfügen, oder bereits erledigte Aufgaben für alle abhaken. Auch können hier relevante Dokumente, die alle Gruppenmitglieder lesen sollen hochgeladen, und bei Bedarf noch kommentiert werden. Für die Dokumentation lässt sich an diesem System auch einfach abgleichen, zu welchen Zeitpunkten die einzelnen Aufgaben abgeschlossen wurden.

\subsection{Github}
Bei Github handelt es sich um einen webbasierten Online-Dienst, der Server für Entwicklungsprojekte mit einer Versionsverwaltung bereitstellt. So können alle Daten nach einer Änderung im Programm wieder hochgeladen und mit einem Kommentar versehen werden. Sollte nach mehreren Veränderungen ein Problem auftreten, kann auf eine ältere Version zurückgegriffen und somit kann der Fehler eingegrenzt werden. Auch kann ein Projekt in Teilabschnitte aufgeteilt werden, damit mehrere Personen unabhängig voneinander daran arbeiten können. Nach der Bearbeitung können diese wieder zusammengefügt werden. Dabei ist erkennbar, welche Änderungen, von wem vorgenommen wurden. So können alle Vorgänge jederzeit verfolgt werden, um eine größtmögliche Übersicht zu gewährleisten. Durch das Kommentieren der Änderungen kann die Nachvollziehbarkeit dieser ebenfalls deutlich gesteigert werden. Des weiteren ist diese Plattform gerade für Unternehmen wie Tinkerforge, die ihren Quellcode als Open-Source anbieten besonders praktisch, da den Nutzern hier alle veröffentlichten Daten direkt zur Verfügung stehen.
\section{Richtlinien zu Erstellung von Schaltplänen und Platinenlayout anhand einem ECAD-Programmpaket}

Das Open Source ECAD-Programm KiCAD ist eine Anwendung zum Erstellen von Schaltplänen und elektronischen Leiterplatten. Mit einem solchen Programm lassen sich die erstellten Schaltpläne vor einer Leiterplattenerstellung auf Verdrahtungsfehler prüfen, um spätere Probleme zu vermeiden.

\subsection{Schaltplan}
Beim Schaltplanentwurf gilt es auf gewisse Regeln zu achten, zu dem ist auf die Übersichtlichkeit des Schaltplans zu achten.
So sollten beispielsweise Filterkondensatoren an der Spannungsversorgung des Mikrocontrollers eingeplant werden, um die Versorgungsspannung zu stabilisieren. Auch ist schon bei dem Entwurf des Schaltplans an das spätere Platinenlayout zu denken. So muss bei der Platzierung der Bauteile darauf geachtet werden, dass die Signalintegrität gewährleistet ist, und es zu keiner Potential Verschiebung kommt.\\
Im Schaltplan wurden die Baugruppen Empfänger, Sender, Hochsetzsteller und die Anschlüsse vom Controller getrennt und so positioniert, dass die beste Übersicht dargestellt wird.\\
Um die einzelnen Bestandteile separat testen zu können, wurden dem Design NULL Ohm Widerstände, an den Verbindungspunkten, hinzugefügt. Außerdem ist es dadurch möglich, eine Zerstörung einzelner Baugruppen durch einen Verdrahtungsfehler bei der Erstinitialisierung zu vermeiden.\\   

\subsection{Platinenlayout}
Beim Entwerfen eines Platinenlayouts gibt es viele Alternativlösungen ein Ergebnis zu erzielen. Somit können alle Bauteile so angeordnet werden, dass alle parallelen Bauteile nebeneinander aufgereiht werden und die in Reihe dazu liegenden Bauteile darunter geordnet sind. So sähe die Platine zwar ähnlich eines Kontaktplanes aus, jedoch ist diese Variante aus Sicht der EMV nicht sonderlich empfehlenswert.\\
Eine andere Gestaltungsmöglichkeit wäre, die Bauteile schon im Schaltbild in Gruppen zusammenzulegen und den Schaltplan auf der Platine exakt nachzubilden. Auch bei dieser Variante ergeben sich gelegentlich Probleme, was die Führung der Leitungen und vor allem den Verlauf der Ströme angeht.\\
Es sollte ein Augenmerk auf den stromführenden Leitungen liegen. Je höher der Strom ist, desto dicker und kürzer ist die Leitung auszulegen, um weniger EMV-Störungen zu erzeugen. Auch sollte die Rückführung (GND) günstigerweise als eigene Leiterschicht ausgeführt werden, um einen großen Leiterquerschnitt zu gewährleisten. So kann bei der Rückführung der Ströme auch das Risiko vermieden werden, durch Bildung von größeren Schleifen, Antennen zu erzeugen. Die GND-Schicht sollte so wenig wie möglich unterbrochen werden, vor allem sind Unterbrechungen quer zur Stromflussrichtung zu vermeiden. Zusätzlich ist zu beachten, dass Kondensatoren, die der Verringerung von störenden Spannungsschwankungen dienen, nahe schaltenden Bauteilen angebracht werden. Die optimale Platzierung ist direkt am VDD oder VCC des ICs, sodass die Leiterbahn vor dem IC mit einem höheren Querschnitt am Kondensator liegt und dann mit leicht verringertem Querschnitt an dem IC angeschlossen ist.






\chapter{Vorbereitung}
Bevor ein erster Entwurf der Schaltpläne durchgeführt werden konnte, mussten Informationen zu den zu verwendenden Bauteilen eingeholt werden, um sicherzustellen, dass diese auch die passenden Betriebsparameter haben und untereinander kompatibel sind.
\section{Recherche der Funktionsweise}
\section{Sender}
Für den Sender (und empfänger) wurde ein auf Pietzomodulen basierendes Kapselgehäuse verwendet. Dabei wurden für den Prototypen mehrere Sender vereschiedener Hersteller bestellt, um Unterschiede der verschiedenpreisigen Bauteile zu ermitteln und festzustellen, welches Preissegment die nötige Qualität für die vorliegende Anwendung erfüllt.

\section{H-Brücke}
Der Lautsprecher wird mit einer Frequenz von 40kHz und einer Amplitude von 20V P-P betrieben. Um ein positives sowie negatives Rechtecksignal zu generieren wird eine H-Brücke verwendet. Sie Abb.x Funktion Block Diagramm vom IC A5950\\
In der Abb. Funktion Block Diagramm vom IC A5950 wird an OUT 1 sowie an OUT 2 eine Stromumkehrung erzielt, durch die Ansteuerung von den MOS-FET von dem Control Logic. Eine genaue Funktionsanalyse ist nicht erforderlich, weil an der H-Brücke nur ein Ultraschallsensor
angeschlossen wird. Wichtig für die H-Brücke ist das sie mit 40kHz schalten kann die am Eingang. Ausgründen der internen Beschaltung und Toleranzen sind am Ausgang nicht die vollen 40kHz zu erwarten, sondern ein Verzug was später im Kapitel X.X behandelt wird.
\section{Empfänger}
Abb.1\\
In der obigen Schaltung(Bildverweis) ist zu sehen dass das ankommende Sinusförmige Signal, verstärkt und in ein digitales Signal umgewandelt wird. Die Schaltung wurde mit einem Hochpassfilter (CR Glied) bestückt bestehend aus C12 und R5 um unerwünschte Signalanteile von unter 40kHz zu unterdrücken. Der Widerstand wurde nach der e24 Reihe, anhand der folgende Berechnung, ausgewählt:
\onehalfspacing \\
\(\displaystyle C=\frac{1}{2*pi*fg*R}\Rightarrow\frac{1}{2*pi*40kHz*C12*100 Ohm}\approx40pF \)
\singlespacing
Die Kapazität des Kondensators C12 wurde an die Grenzfrequenz von 40 kHz und den Widerstand angepasst.\\
Für die Verstärkung der Amplitude so wie der Umwandlung des analog Signals in ein Rechtecksignal mit 40 KHz  dienen die Operationsverstärker des Bausteines TLC272. Die Versorgungsspannung der OPV's von 3,3V wird durch den Kondensator C16 (EMV Störfilter) stabilisiert.\\
Für die Verstärkung der Amplitude ist der Operationsverstärker TLC272 U2B als  nicht invertierender Verstärker geschaltet.
Wenn eine Gleichspannung anliegt, wirkt der Kondensator (C10) in der Operationsverstärkerschaltung als Impedanzwandler, also mit einer Verstärkung von eins, geht nun die Eingangsfrequenz hoch, nimmt der widerstand des Kondensators (C10) ab, somit beginnt der Operationsverstärker auch zu verstärken, und zwar mit zunehmender Verstärkung, bis irgendwann die Impedanz des Kondensators vernachlässigt werden kann und die Verstärkung nur noch durch das Verhältnis der Widerstände beeinflusst wird, R6 ist zudem notwendig um das schwingen der Amplitude zu verhindern, somit kann die Verstärkung mit folgender Formel berechnet werden:
\onehalfspacing \\
\(\displaystyle Vu=R6+R8+\frac{R12}{R6} .\) 
\singlespacing
Die Z-Diode D2 ist für die Spannungsstabilisierung und als Sicherung da.\\
Für die Umwandlung des Analogen Signales in ein Digitales wurde der Operationsverstärker TLC272 U2C als Komparator geschaltet. Beim auftreten von Differenzen zwischen den eingangs Signalen, wechselt der Ausgang des Komparators zwischen Low (0 Volt) auf High (3,3 Volt).

\section{Hochsetzsteller}
Der Hochsetzsteller dient dazu, aus den 5V Versorgungsspannung eine (für die Versuche variable) höhere Spannung für den Sendebetrieb zu schaffen. So kann der Schalldruck der ausgegeben wird erhöht werden(größere Reichweite/größeres Rücksignal)\\
Die Funktionsweise des Hochsetzstellers (Spannungspumpe/Aufwärtswandler/Aufwärtsregler) ist relativ simpel und findet in vielen Bereichen Anwendung. Grundsätzlich wird eine Induktivität in Reihe mit einer Freilaufdiode vor einen Ladecondensator geschaltet. Dieser liegt parallel zum Ausgang. Zwischen der Spule und der Diode ist ein Schalter angeschlossen, der die Spule gegen Masse schaltet. So läd sich die Spule bei Betätigung des Schalters auf (durch den Stromfluss entsteht ein Magnetfeld) und beim Öffnen steigt die Spannung am sekundären Ende der Spule, durch das zusammenbrechende Magnetfeld, an und läd den Kondensator auf. Dieser Vorgang wird wiederholt, bis der Kondensator so weit aufgeladen ist, dass die Ausgangsspannung den gewünschten Wert hält. Dann wird die Schaltfrequenz auf das mindestmaß verringert, um den Wert zu halten. Natürlich ist die mögliche Ausgangsspannung nicht unbegrenzt über das Schaltspiel regelbar, sondern ist auch von den Baugrößen der Bauteile abhängig. Mit einer Induktivität von 10mH und einem Kapazität von 40uF lässt sich die Ausgangsspannung bei 5v Eingangsspannung zwischen 6v und 20v einstellen.


\section{Mikroprozessor}
Der Infineon XMC 1xxx48 gehört zu der Familie der ARM Cortex -M0 Prozessoren und ist ein 32-bit Industrial Microcontroller und wird mit 48MHz externer Clock betrieben. Die 48 steht für die Anzahl der Pins. Der interne Timer läuft mit 96Mhz. Neben bietet der XMC einen 12 bit A/D Wandler, welcher für die Analogmessung eine viel genauere Auflösung bieten kann als ein 8 bit A/D Wandler. Die Betriebsspannung des Prozessors beträgt 3,3V.\\
https://www.infineon.com/cms/en/product/microcontroller/32-bit- industrial-microcontroller- based-
on-arm- cortex-m/32- bit-xmc1000- industrial-microcontroller- arm-cortex- m0/\#\\

\section{Verwendete sowie Dimensionierung der Bauteile}
\subsection{Sender}
Für den Sender (und empfänger) wurden auf Pietzomodulen basierende Kapsellautsprecher verwendet. Dabei wurden für den Prototypen mehrere Kapseln vereschiedener Hersteller bestellt. Dieses geschah um Unterschiede der verschiedenpreisigen Bauteile zu ermitteln und festzustellen, welches Preissegment die nötige Qualität für die vorliegende Anwendung erfüllt. Wichtig sind bei der Überprüfung das nachschwingen der Kapsel, nach Ende des zu sendenden Signals, die Empfindlichkeit auf eingehende Störfrequenzen und die zu erreichende Reichweite mit verschiedenen Signalstärken.

\subsection{Umschaltung}
Da der Sende-, und der Empfangsbetrieb über eine Kapsel laufen sollen und die beiden Betriebsarten mit verschiedenen Spannungspegeln arbeiten, ist eine Umschaltung zwischen den Anschlüssen notwendig.\\
Zu Beginn wurde eine H-Brücke als optionale Lösung des Problems ins Auge genommen. Dafür wurde das IC A5950 verwendet. Wie in Abbildung \ref{fig:} zu sehen ist, kann die H-Brücke eine angeschlossene Last mit einer Wechselspannung versorgen, deren Frequenz über das Signal am Anschluss PHASE vorgegeben werden kann.

Später wurde auf eine Halbbrücke umgeschwenkt, bei der die MOSFETs einzeln ansteuerbar waren. Diese Halbbrücke bietet den Vorteil, dass bei der Ansteuerung der einzelnen MOSFETs gezielt Zwischenzeiten zwischen den Schaltvorgängen eingepflegt werden können, damit ein Abstand zwischen den Schaltflanken entsteht.

%Die Kapsel wird mit einer Frequenz von 40kHz und einer Amplitude von bis zu 20V P betrieben. Um ein positives sowie negatives Rechtecksignal zu generieren wird eine H-Brücke verwendet. Sie Abb.x Funktion Block Diagramm vom IC A5950\\
%In der Abb. Funktion Block Diagramm vom IC A5950 wird an OUT 1 sowie an OUT 2 eine Stromumkehrung erzielt, durch die Ansteuerung von den MOS-FET von dem Control Logic. Eine genaue Funktionsanalyse ist nicht erforderlich, weil an der H-Brücke nur ein Ultraschallsensor
%angeschlossen wird. Wichtig für die H-Brücke ist das sie mit 40kHz schalten kann die am Eingang. Ausgründen der internen Beschaltung und Toleranzen sind am Ausgang nicht die vollen 40kHz zu erwarten, sondern ein Verzug was später im Kapitel X.X behandelt wird.
\subsection{Empfänger}
Die Abbildung \ref{fig:empfängerschaltung} zeigt die Enpfängerschaltung. Durch diese Verschaltung von Operationsverstärkern(OPVs) wird das ankommende Sinusförmige Signal verstärkt und in ein digitales Signal umgewandelt. Die Schaltung wurde mit einem Hochpassfilter (CR Glied) bestückt bestehend aus C12 und R5 um unerwünschte Signalanteile mit Frequenzen, die unter 40kHz liegen, zu unterdrücken. Der Widerstand wurde nach der e24 Reihe, anhand der folgende Berechnung, ausgewählt:
\onehalfspacing \\
\(\displaystyle C12=\frac{1}{2*pi*fg*R}\Rightarrow\frac{1}{2*pi*40kHz*100 Ohm}\approx40pF \)
\singlespacing
Die Kapazität des Kondensators C12 wurde an die Grenzfrequenz von 40 kHz und den Widerstand angepasst, durch nachtregliche Versuche wurde festgestellt dass durch die erhöhung der Kapazität auch die Qualität der Filterung des Signals sich verbessert, in den folgenden Abbildungen\ref{fig:Hochpass 40pF} ,\ref{fig:Hochpass 100nF} sind die unterschiede zusehen. \\
Für die Verstärkung der Amplitude so wie der Umwandlung des analogen Signals in ein Rechtecksignal mit 40 KHz  dienen die Operationsverstärker des Bausteines TLC272. Die Versorgungsspannung der OPV's von 3,3V wird durch den Kondensator C16 (EMV Störfilter) stabilisiert.\\
Für die Verstärkung der Amplitude ist der Operationsverstärker TLC272 U2B als  nicht invertierender Verstärker geschaltet.
Wenn eine Gleichspannung anliegt, wirkt der Kondensator (C10) in der Operationsverstärkerschaltung als Impedanzwandler, also mit einer Verstärkung von eins, geht nun die Eingangsfrequenz hoch, nimmt der widerstand des Kondensators (C10) ab, somit beginnt der Operationsverstärker auch zu verstärken, und zwar mit zunehmender Verstärkung, bis irgendwann die Impedanz des Kondensators vernachlässigt werden kann und die Verstärkung nur noch durch das Verhältnis der Widerstände beeinflusst wird, R6 ist zudem notwendig um das schwingen der Amplitude zu verhindern, somit kann die Verstärkung mit folgender Formel berechnet werden:
\onehalfspacing \\
\(\displaystyle Vu=R6+R8+\frac{R12}{R6} .\) 
\singlespacing
Für die Umwandlung des Analogen Signales in ein Digitales wurde der Operationsverstärker TLC272 U2C als Komparator geschaltet. Beim auftreten von Differenzen zwischen den eingangs Signalen, wechselt der Ausgang des Komparators zwischen Low (0 Volt) auf High (3,3 Volt).

\subsection{Hochsetzsteller}
Der Hochsetzsteller dient dazu, aus den 5V Versorgungsspannung eine (für die Versuche variable) höhere Spannung für den Sendebetrieb zu schaffen. So kann der Schalldruck der ausgegeben wird erhöht werden(größere Reichweite/größeres Rücksignal)\\
Die Funktionsweise des Hochsetzstellers (Spannungspumpe/Aufwärtswandler/Aufwärtsregler) ist relativ simpel und findet in vielen Bereichen Anwendung. Grundsätzlich wird eine Induktivität in Reihe mit einer Freilaufdiode vor einen Ladecondensator geschaltet. Dieser liegt parallel zum Ausgang. Zwischen der Spule und der Diode ist ein Schalter angeschlossen, der die Spule gegen Masse schaltet. So läd sich die Spule bei Betätigung des Schalters auf (durch den Stromfluss entsteht ein Magnetfeld) und beim Öffnen steigt die Spannung am sekundären Ende der Spule, durch das zusammenbrechende Magnetfeld, an und läd den Kondensator auf. Dieser Vorgang wird wiederholt, bis der Kondensator so weit aufgeladen ist, dass die Ausgangsspannung den gewünschten Wert hält. Dann wird die Schaltfrequenz auf das mindestmaß verringert, um den Wert zu halten. Natürlich ist die mögliche Ausgangsspannung nicht unbegrenzt über das Schaltspiel regelbar, sondern ist auch von den Baugrößen der Bauteile abhängig. Mit einer Induktivität von 10mH und einem Kapazität von 40uF lässt sich die Ausgangsspannung bei 5v Eingangsspannung zwischen 6v und 20v einstellen.
\onehalfspacing \\
\(\displaystyle R1=R2*\left(\frac{Vout}{1,23}-1\right)\Rightarrow Vout=\left(\frac{R1}{R2}+1\right)*1,23\) 
\singlespacing

\subsection{Mikroprozessor}
Der Infineon XMC 1xxx48 gehört zu der Familie der ARM Cortex -M0 Prozessoren und ist ein 32-bit Industrial Microcontroller und wird mit 48MHz externer Clock betrieben. Die 48 im Namen des Prozessors steht für die Anzahl der Pins. Der interne Timer läuft mit 96Mhz. Außerdem bietet der XMC einen 12 bit A/D Wandler, welcher für die Analogmessung eine viel genauere Auflösung bieten kann als ein 8 bit A/D Wandler. Die Betriebsspannung des Prozessors beträgt 3,3V.\\
https://www.infineon.com/cms/en/product/microcontroller/32-bit- industrial-microcontroller- based-
on-arm- cortex-m/32- bit-xmc1000- industrial-microcontroller- arm-cortex- m0/\#\\



\section{Entwicklung der Software zum Betrieb des Prototypen}
\subsection{Benötigte Kenntnisse}
Die Variable x=0-3 dient als Index.\\
Um das Programm zu erstellen sollten Kenntnisse z.B. für die Timer CCU4x sowie deren Slices CC40-43 vorhanden sein. Die Funktionen der CCU4x, die Capture Compare Unite ist die Timer/Zählereinheit des Mikrocontrollers, wird in der Reference Manuel beschrieben, siehe dazu Anhang. Um die Timer zu Initialisieren wird die XMC Lib benötigt die eine fertige API mit bringt, siehe Kapitel Besonderheiten der Software, so wird nur noch die API mit den richtigigen Parametern beschrieben. Der Werte der Register mit denen die Zeit gemessen wird müssen sofort zum beginn der Interrupt Service Routine zwischen gespeichert werden damit sichergestelt wird das nicht die Zeit zur Berechung mit gemessen wird.\\

\subsubsection{Durchgeführte Berechnungen}
Auch für die Programmierung waren diverse Berechnungen notwendig. So zum Beispiel musste zur Erzeugung der Ultraschallimpulse ein Pulsweitenmoduliertes Rechteck signal geschaffen werden. Dafür wurde ein Timer der CCU4x auf einen Takt von 40kHz eingestellt. So musste bei einem Timertakt von 96MHz eine Periodendauer von 2400 Takten konfiguriert sein und ein Compare-Wert von 1200 Takten, siehe unten die Berechnung. Im Zählvorgang des Timers wird der Ausgang nach erreichen des Compare-Wertes auf 1 gesetzt, und nach erreichen der Periodendauer wieder auf 0 zurückgesetzt. Dadurch ergibt sich eine Periodendauer von 25us , was einer Frequenz von 40kHz entspricht.\\
Auch zur Erfassung der Zeit, die vergeht bis das Echo des Ultraschall-Impulses zurück kommt wird über einen Timer der CCU4x erfasst. 
\onehalfspacing \\ \\
\(\displaystyle Periodendauer=\frac{96MHz}{40kHz} = 2400 \)  \  \  \    \(\displaystyle Compare-Wert=\frac{2400}{2} = 1200 \) 
\singlespacing

Siehe Kapitel Messungen am Ausgang des Controller für das PWM Signal sowie die Einstellung der einzelnen Timer, Anhang PWM Configuration sowie Timer Configuration.

\subsection{Quellcodeentwurf}

\textbf{Programmstruktur:}
Anstatt alles in der Distance US main.c ,siehe Abbildung \ref{fig:main.c1}, an Programmcode zu verfassen was bei sehr komplexen Programmen schnell zu Unübersichtlichkeit führt hat das Auslagern den Vorteil das der Quellcode Logisch getrennt werden kann und so einer verschlankerung des Codes mit sich bringt. 
Somit stehen in der Main  vor allem die Aufrufe der verschiedenen benötigten Funktionen. So sieht man in der Main jetzt deutlich, welche Funktionen beim Starten initialisiert werden, und welche Unterprogramme regelmäßig aufgerufen werden. Auch vereinfacht diese Struktur gerade bei Prototypen das Testen der Funktion, so kann im Falle einerfehlerhaften Funktion einfach der Aufruf auskommentiert werden um zu testen, ob der Fehler wirklich von der Funktion herrührt. Dadurch müssen nicht etliche Zeilen Programmcode der Funktion auskommentiert werden, wodurch schnell Fehler entstehen könnten, durch übriggebliebene Zeichen, oder gar beim entfernen der Auskommentierung gelöschte Zeichen.\\
\begin{minipage}{1\textwidth}
\begin{lstlisting}
#include <stdio.h>
#include <stdbool.h>
#include "bricklib2/logging/logging.h"
#include "bricklib2/bootloader/bootloader.h"
#include "communication.h"
/****Eigene Include Dateien*******/
#include "configs/config.h"
#include "system_timer/system_timer.h"
#include "a16pt.h"
int main(void)
{ 
	logging_init(); 
	logd("Start Distance US V2 Bricklet/n/r");  	//For the Debugmodus
	communication_init(); 					//Function call
	a16pt_init(); 								//Function call	
	while(true)
	{
		a16pt_tick(); 						//Function call
		bootloader_tick(); 					//Function call
		communication_tick(); 				//Function call
		
	}
}
\end{lstlisting}
\captionof{figure}{Die main.c des Distance US}
\label{fig:main.c1}
\end{minipage}\\
Um die FUnktionsaufrufe zu verstehen muss die Abbildung \ref{fig:a16pt.h}: config a16pt.h näher betrachtet werden.
In der der a16pt.h werden die Funktionen definiert die dann in der main.c aufgerufen werden und die Funktionsanweisungen stehen dafür in der a16pt.c. \ref{fig:a16pt.c}\\
\begin{minipage}{1\textwidth}
\begin{lstlisting}
#ifndef A16PT_H
#define A16PT_H
#include <stdint.h>
void a16pt_init(void);				//Functional definition
void a16pt_tick(void); 			//Functional definition
uint16_t a16pt_get_distance(void); //Functional definition
#endif
\end{lstlisting}
\captionof{figure}{config a16pt.h}
\label{fig:a16pt.h}
\end{minipage}\\
\\
\textbf{Init Aufruf:}
Auch in der \ref{fig:a16pt.c}void a16pt-init(void) gibt es weitere Funktionsaufrufe wo z.B. die PWM oder der Externe Interrupt Initialisiert werden.

\begin{minipage}{1\textwidth}
\begin{lstlisting}
void a16pt_init(void)
{

/*****************Externe_Interrupt*******************/

	eru_init(eru_port);

/************PWM_Init*****************************/

	XMC_CCU4_Init(CCU41, XMC_CCU4_SLICE_MCMS_ACTION_TRANSFER_PR_CR_PCMP);
	XMC_CCU4_StartPrescaler(CCU41);

	ccu4_pwm_init(pwm_port_0,cc40, period_1);	//P4_4
	ccu4_pwm_set_duty_cycle( cc40, compare_1);

	ccu4_pwm_init(pwm_port_1,cc42, period_0);	//P4_6
	ccu4_pwm_set_duty_cycle( cc42, compare_0);
.
.
.
.
\end{lstlisting}
\captionof{figure}{Ein Teilausschnitt von der a16pt.c mit dem Init Aufruf }
\label{fig:a16pt.c}
\end{minipage}



\textbf{Interrupt Aufruf:}
In der Abbildung \ref{fig:a16pt.c1} :a16pt.c Interrupt Aufruf, werden die für die Entfernungsmessung notwendigen Funktionen und die Interrupt anweisungen, in dem Fall die IRQ21, abgearbeitet außerdem werden die Timer Synchron abgeschaltet und aus experementellen gründen wurde ein weitere Impuls generiert um zu beobachten wie sich das nachschwingen verhält bei einer längeren Kurzschlusszeit an der Ultraschallkapsell. Die IRQ wird auch als Interrupt bezeichnet und wird von der Hardware oder von der Software ausgelöst.
\\
\begin{minipage}{1\textwidth}
\begin{lstlisting}

/*************Interrupt_Funktionen****************/

void IRQ_Hdlr_21(void) // Compare Interrupt counter 10
{

	// Disable IRQs so we can't be interrupted
	__disable_irq();

	// Set CCU trigger to low, otherwise ccu counter is restarted
	XMC_SCU_SetCcuTriggerLow(XMC_SCU_CCU_TRIGGER_CCU41);

	// Stop slice 2
	XMC_CCU4_SLICE_StopClearTimer(CCU41_CC40);

	// For slice 1 we wait until PWM is run through (to get exactly 10 pwm peaks on P4_4 and P4_6)
	while(XMC_CCU4_SLICE_GetTimerValue(CCU41_CC42) > compare_1) {

		__NOP();
	}
	
	//new pin configuration
	const XMC_GPIO_CONFIG_t pin_out_config	= {
			.mode                = XMC_GPIO_MODE_OUTPUT_PUSH_PULL,
			.output_level        = XMC_GPIO_OUTPUT_LEVEL_HIGH,
		};

	 XMC_GPIO_Init(P4_6, &pin_out_config);
	//Creat a high impulse
	for(s=0; s<50; s++)
		{
			__NOP();
		}
	// Stop slice 0
	XMC_CCU4_SLICE_StopClearTimer(CCU41_CC42);
	
	//pin configuration back to the PWM-Mode
	const XMC_GPIO_CONFIG_t gpio_out_config1	= {
		.mode                = XMC_GPIO_MODE_OUTPUT_PUSH_PULL_ALT9,
		.input_hysteresis    = XMC_GPIO_INPUT_HYSTERESIS_STANDARD,
		.output_level        = XMC_GPIO_OUTPUT_LEVEL_LOW,
	};

	XMC_GPIO_Init(P4_6, &gpio_out_config1);
	// Enable IRQs again
	__enable_irq();


}
\end{lstlisting}
\captionof{figure}{Ein Teilausschnitt von der a16pt.c mit dem Interrupt Aufruf}
\label{fig:a16pt.c1}
\end{minipage}

%\newpage
%\begin{minipage}{1\textwidth}Eventuell wo anders hin
%\begin{struktogramm}(120,75)
%\forever
%\assign{\#include aufrufe}
%\while[8]{int main (void)}
 %\sub{Logging init()}
 %\sub{logd ("start Distance us v2 Bricklet")}
 %\sub{Communication init()}
 %\sub{a16pt init()}
%\while[8]{while (1)}
 %\sub{a16pt\_tick()}
 %\sub{bootloader\_tick()}
% \sub{Communication\_tick()}
%\whileend
%\whileend
%\foreverend

%  \ifthenelse{10}{4}{Bedingung 1}{ja}{nein}
%    \ifthenelse{6}{6}{Bedingung 2}{ja}{nein}
%      \assign{Anweisungsblock 1}
%    \change
%      \assign{Anweisungsblock 2}
%    \ifend
%  \change
%    \assign{Anweisungsblock 3}
%  \ifend
%\sub{bla}
%\end{struktogramm}
%\captionof{figure}{Struktogramm der main}\label{fig:Struktogramm der main}
%\end{minipage}

%\newpage
%\begin{figure}[H]
%\includegraphics[width=1.0\textwidth]{Struktogramme/a16pt.png}\caption{Struktogramm der a16.pt}\label{fig:Bild2}
%\end{figure}




\chapter{Messungen und Auswertung der Ergebnisse}
\label{Messungen}
Für die Messungen wurden Laufe des Projekts zwei Versionen an Prototyp-Platinen entworfen, an denen Messungen und Verbesserungen vorgenommen wurden. 
\section{Prototyp 1}
Bei dem ersten Prototyp wurden die Sendereinheit und die Empfängereinheit auf getrennten Platinen aufgebaut. So bestand die Möglichkeit, den Senderkreis und den Empfängerkreis getrennt zu untersuchen, ohne dass sich elektrische Signale der beiden Schaltkreise überlagern konnten.

\subsection{Senderkreis}
Zu erst wurden Signale direkt an der CPU gemessen, um sicher zu stellen, dass die Einstellungen im Programm auch die gewünschten Ausgaben zur Folge haben, und keine Gefärdung der Bauteile entsteht.
Um das Signal für die Entfernungsmessung zu generieren wurde der Mikrocontroller so programmiert, dass zehn Impulse mit einer Frequenz von 40kHz ausgegeben werden. Danach erfolgt eine Pause, um das zurückkehrende Signal abzuwarten und auszuwerten.\\
\begin{minipage}{0.5\textwidth}
\includegraphics[width=1\textwidth%, draft
]{Abbildungen/MessungenP1/PWM-von-der-cpu.png}
\captionof{figure}{PWM-Burst auf 40kHz Basis an der CPU}
\label{fig:pwm-burst}
\end{minipage}
\begin{minipage}{0.5\textwidth}
\includegraphics[width=1\textwidth%, draft
]{Abbildungen/MessungenP1/PWM-ausgabe-mit-Hi-Side.png}
\captionof{figure}{PWM Ausgabe über einen Hi-Side}
\label{fig:HiSide}
\end{minipage}
In der Abbildung \ref{fig:pwm-burst} ist zu sehen, dass der gewünschte Burst aus zehn Impulsen mit einer Periodendauer von jeweils 25us vom Mikrocontroller generiert wurde. Diese Messung wurde auch vorgenommen, um zu überprüfen, wie sich das Signal durch die eingesetzten Bauteile verändert.\\
Die Abbildung \ref{fig:HiSide} zeigt, wie das Ausgangssignal nach einer Hi-Side aussieht. So wird zwar im Takt des PWM-Signals geschaltet, allerdings fehlt es an einem Gegenpool, um das Potential in den Schaltpausen wieder auf Null zu ziehen. Dadurch bleibt die Spannung während des Schaltens immer auf einem erhöhten Pegel und sinkt erst nach Ende des PWM-Signals langsam ab. Dadurch kann natürlich keine vernünftige Ausgabe am Lautsprecher erzeugt werden, denn ohne deutliche Potentialunterschiede kann dieser auch nicht in Schwingungen versetzt werden. Der ausgegebene Schalldruck würde maximal für kürzeste Entfernungsmessungen reichen, wenn überhaupt und dann würde das zurückkommende Signal noch von der abklingenden Spannung des Hi-Side überlagert. Somit ist dieser Aufbau nicht operabel.\\
Um die Spannung nicht nur auf einen Hi-Pegel, sondern auch auf einen LOW-Pegel schalten zu können wurde danach auf eine Halbbrücke gewechselt. Mit dieser lässt sich der Ausgang, über zwei durch das PWM-Signal gesteuerte MOSFETs, sauber auf Hi- oder LOW-Pegel schalten. %Bei der ersten, einfach gesteuerten Version, entstand das Problem, dass beim Schalten der Halbbrücke, beide MOSFETs gleichzeitig geschaltet haben. Dieses klingt zwar nicht nach einem Problem, doch wird es durch die technischen Gegebenheiten zu einem, denn bei den meisten elektronischen Schaltern verläuft der Einschaltprozess deutlich schneller, als der Ausschaltprozess. Wenn also zwei Bauteile gleichzeitig schalten, hat das einschaltende Bauteil schneller eingeschaltet, als das ausschaltende Bauteil ausgeschaltet. Dadurch entstehen bei jedem Schaltvorgang Kurzschlüsse. Auch wenn diese nur für Nanosekunden bestehen, bevor sie wieder unterbrochen werden, ist davon auszugehen, dass zu den dadurch entstehenden Störungen auch Bauteile zerstört werden.\\ Somit wurde auf eine voll gesteuerte Halbbrücke gewechselt und es wurden zwei PWM-Signale moduliert, bei denen eines invertiert und die Flanken derart verschoben waren, dass die angesteuerten MOSFETs nie Kurzschlüsse schalten können.
Mit der verwendeten Halbbrücke ergab sich die Abbildung \ref{fig:Halfbridge}\\
\begin{minipage}{0.5\textwidth}
\includegraphics[width=1\textwidth%, draft
]{Abbildungen/MessungenP1/PWM-Nach-der-Halbbrucke.png}
\captionof{figure}{PWM Ausgabe über eine Halbbrucke}
\label{fig:Halfbridge}
\end{minipage}
\begin{minipage}{0.5\textwidth}
\includegraphics[width=1\textwidth%, draft
]{Abbildungen/MessungenP1/PWM-Nach-der-Halbbrucke-mit-LS.png}
\captionof{figure}{Ausgabe der PWM an der Ultraschallkapsel}
\label{fig:Senderausgabe}
\end{minipage}
Es zeigt sich, dass das Signal nach der Erweiterung auf eine Halbbrücke wieder wie das von der CPU ausgegebene PWM-Signal \ref{fig:pwm-burst} aussieht, nur dass die Amplitude wie geplant höher ausfällt. Somit kann die Höhe der Amplitude über die Spannungspumpe variiert werden um die Stärke des ausgegebenen Signals zu verändern, ohne die CPU durch die höhere Spannung zu beschädigen. Wie in der Abbildung \ref{fig:Senderausgabe} zu entnehmen ist, entstehen durch die angeschlossene Ultraschallkapsel höhere Spannungsimpulse im Einschaltmoment.

\subsection{Empfängerkreis}
Die Platinen des Sender- und Empfängerkreises wurden gemeinsam auf einer Halterung montiert, so dass die Ultraschallkapseln zum senden und empfangen der Signale nebeneinander befestigt werden konnten. Ziel war es, durch verschieben eines Hindernisses die Signaländerungen an den Platinen beobachten zu können, ohne die gesamten Messaufbauten bewegen zu müssen. \\
\begin{minipage}{0.5\textwidth}
\includegraphics[width=1\textwidth%, draft
]{Abbildungen/MessungenP1/Signal-Empfang.png}
\captionof{figure}{Signal Empfang}
\label{fig:Empfang am LS}
\end{minipage}
\begin{minipage}{0.5\textwidth}
\includegraphics[width=1\textwidth%, draft
]{Abbildungen/MessungenP1/Signal-nach-Verstarkung.png}
\captionof{figure}{Signal nach Verstärkung}
\label{fig:Verstaerkung}
\end{minipage}
%\begin{minipage}{0.5\textwidth}
%\includegraphics[width=1\textwidth%, draft
%]{Abbildungen/MessungenP1/Signal-nach-der-Filterung.png}
%\captionof{figure}{Signal nach der Filterung}
%\label{fig:Filterung}
%\end{minipage}
Die Abbildung \ref{fig:Empfang am LS} zeigt das Signal, das direkt am Empfänger zu messen war. Hier sind verschiedene vorerst nicht zuordenbare Signale zu sehen. Allein aus diesem Bild lässt sich aber keine Aussage zu den Signalen machen. Fest steht nur, dass ebenfalls Signale die nicht der gewünschten Frequenz entsprechen, vom Empfänger aufgenommen werden. Dies gilt es natürlich schnellst möglich auszumerzen, um unerwünschte Störungen zu vermeiden.
Die Abbildungen \ref{fig:Verstaerkung} und \ref{fig:Verstaerkung2} zeigen den Verlauf des Signals nach der Filterung und Verstärkung in zwei verschiedenen Zeitauflösungen. Dabei entspricht \ref{fig:Verstaerkung} den ersten drei Kästchen von \ref{fig:Verstaerkung2} und dient um darzustellen, dass die Verstärkung eine maximale Aussteuerung von 3,3V nicht überschreitet.
%In der Abbildung \ref{fig:Filterung} ist das Signal nach dem eingebauten Hochpassfilter zu sehen.\\

\begin{minipage}{0.5\textwidth}
\includegraphics[width=1\textwidth%, draft
]{Abbildungen/MessungenP1/Signal-nach-Verstarkung2.png}
\captionof{figure}{Signal nach Verstärkung2}
\label{fig:Verstaerkung2}
\end{minipage}
\begin{minipage}{0.5\textwidth}
\includegraphics[width=1\textwidth%, draft
]{Abbildungen/MessungenP1/Signal-nach-Komparator.png}
\captionof{figure}{Signal nach Komparator}
\label{fig:Komparator}
\end{minipage}\\
Nach dem das Signal den Komparator passiert hat, ergibt sich das Bild wie in Abbildung \ref{fig:Komparator} zu sehen ist. Bei einem Vergleich mit dem Signal nach der Verstärkung \ref{fig:Verstaerkung2} wird sichtbar, dass der Komparator nur Signale, die über seinem Schwellwert liegen, durchschaltet. Die Aufteilung in zwei Signalblöcke in den Abbildungen kommt daher, dass der erste Block das Signal der Sender-Kapsel ist, das direkt beim Senden seitlich auf die Empfänger-Kapsel abgestrahlt wurde. Der zweite Block ist bereits das Echo, das vom 20cm entfernten Hindernis zurückgeworfen wurde.\\
%Nachfolgend wurden die Signalverläufe mit verschiedenen Ultraschallkapseln aufgenommen, um vergleichen zu können, wie die Signalqualität bei verschiedenen Produkten schwankt. Dabei wurde die Amplitude von 4,6V für das Sendersignal und die Verstärkung des Empfängers auf ein Minimum eingestellt.\\
%\begin{minipage}{0.5\textwidth}
%\includegraphics[width=1\textwidth%, draft
%]{Abbildungen/MessungenP1/EKULIT1,5m.png}
%\captionof{figure}{EKULIT 1,5m}
%\label{fig:EKULIT1,5m}
%\end{minipage}
%\begin{minipage}{0.5\textwidth}
%\includegraphics[width=1\textwidth%, draft
%]{Abbildungen/MessungenP1/MURATAr1,5m.png}
%\captionof{figure}{MURATA reciver 1,5m}
%\label{fig:MURATA reciver 1,5m}
%\end{minipage}
%\begin{minipage}{0.5\textwidth}
%\includegraphics[width=1\textwidth%, draft
%]{Abbildungen/MessungenP1/MURATAs1,5m.png}
%\captionof{figure}{MURATA sender 1,5m}
%\label{fig:MURATA sender 1,5m}
%\end{minipage}
%\begin{minipage}{0.5\textwidth}
%\includegraphics[width=1\textwidth%, draft
%]{Abbildungen/MessungenP1/MURATAsr1,5m.png}
%\captionof{figure}{MURATA sender + reciver 1,5m}
%\label{fig:MURATA sender+reciver}
%\end{minipage}\\
%Bei den Ultraschallkapseln von EKULIT zeigte sowohl eine Verpoolung der Anschlüsse, als auch ein Tausch von Sender- und Empfängerkapsel keinen Unterschied. Bei den Ultraschallkapseln von MURATA sind bei der Wahl der Kapseln dagegen deutliche Unterschiede zu sehen. Werden nur die Sender-Kapseln als Sender und Empfänger verwendet, entsteht das in Abbildung \ref{fig:MURATA sender 1,5m} zu sehende Signal. Die Breite des eingehenden Echo Signals ist deutlich geringer, als bei den anderen verwendeten Kombinationen. Am besten sind die Signale bei der Kombination aus Sender und Empfänger von MURATA, Abbildung \ref{fig:MURATA sender+reciver} oder bei den EKULIT-Kapseln, Abbildung \ref{fig:EKULIT1,5m}.

\section{Prototyp 2}
Bei der zweiten Prototyp-Version wurden der Sender- und der Empfängerkreis auf einer Platine aufgebaut und es wurde nur noch eine Ultraschallkapsel für beide Anwendungen vorgesehen.
Um bei diesem Aufbau, einen fehlerfreien Betrieb der verwendeten voll gesteuerten Halbbrücke sicherzustellen, wurden durch den Mikrocontroller zwei getrennte PWM-Signale generiert, die wie in Abbildung \ref{fig:PWMs} zu sehen ist, durch Lücken getrennt sind. So ist sichergestellt, dass auch trotz Verzögerungen im Schaltbetrieb der Halbleiter, keine Kurzschlüsse entstehen können.\\
% Denn es darf nie außer Acht gelassen werden, dass die Einschaltzeiten von MOSFETs meistens kürzer sind, als die Ausschaltzeiten (letztere sind bis zu dreimal so lang).
\begin{minipage}{0.75\textwidth}
\includegraphics[width=1\textwidth%, draft
]{Abbildungen/MessungenP2/Zwei_PWMs_von_der_CPU.PNG}
\captionof{figure}{Verlauf der zwei generierten PWMs für den Betrieb der voll gesteuerte Halbbrücke}
\label{fig:PWMs}
\end{minipage}\\
Nachdem dieser Betrieb sichergestellt war, wurden Messungen am Verstärker (obere Linie), und am Komparator (untere Linie) vorgenommen. Dabei wurde die Verstärkung so eingestellt, dass unerwünschte Störungen gerade so nicht vom Komparator weitergegeben wurden. Die Spannung für den Sendebetrieb wurde für die Versuche zwischen 5V und 20V variiert, um betrachten zu können, wie sich das auf die Reichweite und Genauigkeit der Messungen auswirkt. Als Hindernis wurde bei allen Versuchen eine glatte Holzplatte der Maße 40x60cm verwendet und in einem Abstand von ein bis fünf Metern von der Ultraschallkapsel aufgestellt.In den Abbildungen \ref{fig:5v1m} bis \ref{fig:5v4m} sind die Ergebnisse einer Messreihe mit einer Spannung von 5V für den Sendebetrieb dargestellt. Die Ansicht wurde so eingestellt, dass zwei Sendeimpulse zu sehen sind. Dadurch wird deutlicher, welches die Sende Impulse sind, und welches die von der Entfernung abhängigen Echos sind.\\
\begin{minipage}{0.5\textwidth}
\includegraphics[width=1\textwidth%, draft
]{Abbildungen/MessungenP2/5V/1mb.PNG}
\captionof{figure}{Signalverlauf bei 5V auf 1m Abstand}
\label{fig:5v1m}
\end{minipage}
\begin{minipage}{0.5\textwidth}
\includegraphics[width=1\textwidth%, draft
]{Abbildungen/MessungenP2/5V/2mb.PNG}
\captionof{figure}{Signalverlauf bei 5V auf 2m Abstand}
\label{fig:5v2m}
\end{minipage}
\begin{minipage}{0.5\textwidth}
\includegraphics[width=1\textwidth%, draft
]{Abbildungen/MessungenP2/5V/3mb.PNG}
\captionof{figure}{Signalverlauf bei 5V auf 3m Abstand}
\label{fig:5v3m}
\end{minipage}
\begin{minipage}{0.5\textwidth}
\includegraphics[width=1\textwidth%, draft
]{Abbildungen/MessungenP2/5V/4mb.PNG}
\captionof{figure}{Signalverlauf bei 5V auf 4m Abstand}
\label{fig:5v4m}
\end{minipage}
Bei den Abbildungen ist zu sehen, dass das Echo-Signal mit zunehmender Entfernung immer schwächer wird. Bei einer Entfernung von fünf Metern (Abbildung \ref{fig:5v4m}) wird das Echo-Signal so schwach, dass die Signalstärke nach dem Komparator nicht mehr für eine eindeutige Auswertung über den Mikrokontroller ausreicht. Nachfolgend sind die Abbildungen einer Messreihe mit verschiedenen Spannungseinstellungen für den Sendebetrieb zu sehen. Anhand dieser Messreihe soll dargestellt werden, welchen Einfluss die eingestellte Spannung im Sendebetrieb auf die Reichweite des Ultraschallsignals hat. Für die Darstellung wurden die Messungen bei 5 Meter Abstand ausgewählt.\\
\begin{minipage}{0.5\textwidth}
\includegraphics[width=1\textwidth%, draft
]{Abbildungen/MessungenP2/5V/5m.PNG}
\captionof{figure}{Signalverlauf bei 5V auf 5m Abstand}
\label{fig:5v5m2}
\end{minipage}
\begin{minipage}{0.5\textwidth}
\includegraphics[width=1\textwidth%, draft
]{Abbildungen/MessungenP2/10V/5mb.PNG}
\captionof{figure}{Signalverlauf bei 10V auf 5m Abstand}
\label{fig:10v5m}
\end{minipage}
\begin{minipage}{0.5\textwidth}
\includegraphics[width=1\textwidth%, draft
]{Abbildungen/MessungenP2/15V/5mb.PNG}
\captionof{figure}{Signalverlauf bei 15V auf 5m Abstand}
\label{fig:15v5m}
\end{minipage}
\begin{minipage}{0.5\textwidth}
\includegraphics[width=1\textwidth%, draft
]{Abbildungen/MessungenP2/20V/5mb.PNG}
\captionof{figure}{Signalverlauf bei 20V auf 5m Abstand}
\label{fig:20v5m}
\end{minipage}
Bei Vergleich der Abbildungen \ref{fig:5v5m2} und \ref{fig:10v5m} mit den Abbildungen \ref{fig:15v5m} und \ref{fig:20v5m}  ist zu sehen, dassbei einem Abstand von 5 Metern erst bei einer Sendespannung von über 10V am Komparator ein über den Mikrokontroller auswertbares Signal vorhanden ist. 









%\begin{minipage}{0.5\textwidth}
%\includegraphics[width=1\textwidth%, draft
%]{Abbildungen/MessungenP2/.PNG}
%\captionof{figure}{}
%\label{fig:}
%\end{minipage}
\section{Fazit aus den Ergebnissen für den Auftraggeber}
Aus den Versuchen und Messungen lassen sich mehrere Aussagen treffen. \\
Als erstes, eine Ultraschall-Entfernungsmessung ist mit wenigen Bauteilen, sowohl als Zwei-Kapsel-Variante, als auch als Ein-Kapsel-Variante durchführbar. Bei der Ein-Kapsel-Variante ist darauf zu achten, dass der Verstärker eine ausreichende Spannungsfestigkeit besitzt, um nicht durch das Sendersignal gestört zu werden. Auch ist wichtig, dass bei der Erzeugung des PW-Modulierten Ausgangssignals die MOSFETs so angesteuert werden, dass den MOSFETs zwischen den Schaltsignalen genug Zeit gegeben wird, um die Schaltzustände zu erreichen. Vergleiche Abbildung \ref{fig:PWMs}. Dadurch lassen sich Kurzschlüsse an dieser Stelle vermeiden.\\
Eine Auswertung des Echo-Signals ist prinzipiell sowohl digital, als auch analog möglich. bei der digitalen Auswertung muss die Zeit erfasst werden, die zwischen dem Senden des PWM-Signals und dem Empfangen des Echo-Signals vergeht. Die errechnete Strecke ist zu halbieren, da die vergangene Zeit sowohl den Hin-, als auch den Rückweg beinhaltet. Bei der analogen Auswertung besteht zwar die Möglichkeit über einen Frequenzvergleich auch Signale mit kleinerer Echo-Amplitude zu erkennen und auszuwerten, allerdings beinhaltet dieses Vorgehen einen deutlich höheren Programmieraufwand.\\
Bei der Berechnung der Zeiten muss berücksichtigt werden, dass das eingehende Echo-Signal die Ultraschallkapsel langsam in Schwingungen versetzt. Die ersten eintreffenden Schwingungen eines PWM-Signals erzeugen kleinere Spannungssignale als die darauf folgenden. Außerdem schwingt die Ultraschallkapsel auch nach Ende des eingehenden Signals noch etwas nach, was zur Folge hat, dass das analoge Abbild des gesendeten PWM-Signals leicht versetzt und etwas verlängert wirkt. All diese Faktoren müssen für eine genauere Berechnung der Strecke, die das Signal zurück gelegt hat, berücksichtigt werden. Allein durch die Tatsache, dass das empfangene Signal an der Ultraschallkapsel derart verändert wird, ist eine absolut exakte Messung nicht möglich. Eine Beschränkung des Fehlers auf einzelne Zentimeter ist aber realisierbar.\\
%Eddy
%Die Problematik ist nun das die Zusammenschaltung von dem Empfänger mit dem Sender was zur Folge hat das ein Ausgang vom A5950 an Masse angeschlossen ist und somit ein Kurzschluss erzeugt wird was beim wechsel von der positiven zur negativen Amplitude geschieht.
%Um eine bessere Aussage zu treffen wurde eine H-Side Schaltung aufgebaut, siehe dazu das Schaltbild x.X, das IC A5950 wurde entfernt. Die H-Side konnte nur den Ultraschallsensor nicht gegen Masse schalten was zur Folge hatte das wir ein nach schwingen am Oszilloskop messen konnten siehe „Kaptiel Messung bild x.x“ 
%Um die Schwierigkeit zu lösen benötigten wir eine Halbbrücke mit einer separaten Ansteuerung ohne eine interne Logik. Somit wurde eine Halb-Brücke aufgebaut was die H-Side Schaltung ersetzt. Die Ansteuerungslogik wurde vom Prozessor XMC 1xxx48 bewerkstelligt. 
%Filterschaltung :
%Durch nachträgliche Versuche wurde festgestellt, dass durch das erhöhen der Kapazität auch die Qualität der Filterung des Signals sich verbessert, in den folgenden Abbildungen\ref{fig:Ohne Filter}, \ref{fig:Hochpass 40pF} ,\ref{fig:Hochpass 100nF} sind die Unterschiede zusehen.
%
%Anhand der Resultate wurde eine zweite Board Version erstellt mit der änderung an der Senderschaltung, Anhang Board V2.

\chapter{Reflektion über den Projektablauf}
Nach Erhalt der Aufgabenbeschreibung war das Bild, das man sich von den bevorstehenden Aufgaben machte, doch sehr anders, als die Abeiten später aussahen. So war mit der Programmierung, auf Grund des doch deutlich komplexeren Mikrocontrollers, ein enormer Lernaufwandt verbunden. Die Recherche der Hardwarebauteile und das Erstellen der Schaltpläne und Platinen hingegen verlief ähnlich den Erwartungen.

Eddy: Die Einarbeitung in die Tinkerforge verwendeten Programme, KiCad und GitHub, erforderten genausoviel Aufmerksamkeit wie die Programmierung trotz der Einarbeitungsphasen war stest konstant ein Lernerfolg zu verbuchen und konnten bis zum ende des Projektes uns neues Wissen sowie Fähigkeiten erwerben.
\chapter{Anhänge}
Abbildungen der Messreihen mit verschiedenen Spannungen bei Abständen von ein bis fünf Metern.
\begin{minipage}{0.5\textwidth}
\includegraphics[width=1\textwidth%, draft
]{Abbildungen/MessungenP2/5V/1m.PNG}
\captionof{figure}{Signalverlauf bei 5V auf 1m Abstand}
\end{minipage}
\begin{minipage}{0.5\textwidth}
\includegraphics[width=1\textwidth%, draft
]{Abbildungen/MessungenP2/10V/1m.PNG}
\captionof{figure}{Signalverlauf bei 10V auf 1m Abstand}
\end{minipage}
\begin{minipage}{0.5\textwidth}
\includegraphics[width=1\textwidth%, draft
]{Abbildungen/MessungenP2/5V/2m.PNG}
\captionof{figure}{Signalverlauf bei 5V auf 2m Abstand}
\end{minipage}
\begin{minipage}{0.5\textwidth}
\includegraphics[width=1\textwidth%, draft
]{Abbildungen/MessungenP2/10V/2m.PNG}
\captionof{figure}{Signalverlauf bei 10V auf 2m Abstand}
\end{minipage}
\begin{minipage}{0.5\textwidth}
\includegraphics[width=1\textwidth%, draft
]{Abbildungen/MessungenP2/5V/3m.PNG}
\captionof{figure}{Signalverlauf bei 5V auf 3m Abstand}
\end{minipage}
\begin{minipage}{0.5\textwidth}
\includegraphics[width=1\textwidth%, draft
]{Abbildungen/MessungenP2/10V/3m.PNG}
\captionof{figure}{Signalverlauf bei 10V auf 3m Abstand}
\end{minipage}
\begin{minipage}{0.5\textwidth}
\includegraphics[width=1\textwidth%, draft
]{Abbildungen/MessungenP2/5V/4m.PNG}
\captionof{figure}{Signalverlauf bei 5V auf 4m Abstand}
\end{minipage}
\begin{minipage}{0.5\textwidth}
\includegraphics[width=1\textwidth%, draft
]{Abbildungen/MessungenP2/10V/4m.PNG}
\captionof{figure}{Signalverlauf bei 10V auf 4m Abstand}
\end{minipage}
\begin{minipage}{0.5\textwidth}
\includegraphics[width=1\textwidth%, draft
]{Abbildungen/MessungenP2/5V/5m.PNG}
\captionof{figure}{Signalverlauf bei 5V auf 5m Abstand}
\end{minipage}
\begin{minipage}{0.5\textwidth}
\includegraphics[width=1\textwidth%, draft
]{Abbildungen/MessungenP2/10V/5m.PNG}
\captionof{figure}{Signalverlauf bei 10V auf 5m Abstand}
\end{minipage}

%\begin{minipage}{1\textwidth}
%\begin{tabularx}{\textwidth}{p{0.32\textwidth}|r|r|r|r}
%Entfernung [m]& Zeit bis Anfang Echo [ms] & Zeit bis Ende Echo [ms] & Errechnete Entfernung Anfang [m] & Errechnete Entfernung Ende [m]\\
%\hline
%1 & 6,11 & 7,23 & 1,0485 & 1,2407\\
%1,5 & 9,02 & 9,89 & 1,5478 & 1,6971\\
%2 & 11,97 & 12,76 & 2,0541 & 2,1896\\
%2,5 & 14,88 & 15,5 & 2,5534 & 2,659\\
%3 & 17,84 & 18,2 & 3,06134 & 3,1231\\
%3,5 & 20,8 & 21,11 & 3,569 & 3,6225\\
%4 & 23,71 & 23,81 & 4,0686 & 4,0858\\
%4,5 & 26,61 & 26,73 & 4,5663 & 4,5869\\
%5 & 29,59 & 29,68 & 5,0776 & 5,0961\\
%\end{tabularx}
%\captionof{table}{Entfernungsmessung bei 5V Sendespannung}
%\label{tab:Entfernungsmessung5V}
%\begin{tabularx}{\textwidth}{p{0.12\textwidth}|r|r|r|r}
%Entfernung [m]& Zeit bis Anfang Echo [ms] & Zeit bis Ende Echo [ms] & Errechnete Entfernung Anfang [m] & Errechnete Entfernung Ende [m]\\
%\hline
%1 & 6,07 & 7,37 & 1,0416 & 1,2647\\
%1,5 & 8,99 & 10,02 & 1,5427 & 1,7194\\
%2 & 11,94 & 12,9 & 2,0489 & 2,2136\\
%2,5 & 14,83 & 15,7 & 2,5448 & 2,6941\\
%3 & 17,8 & 18,48 & 3,0545 & 3,1712\\
%3,5 & 20,7 & 21,36 & 3,5521 & 3,6654\\
%4 & 23,62 & 24 & 4,0532 & 4,1184\\
%4,5 & 26,55 & 26,9 & 4,5560 & 4,6160\\
%5 & 29,5 & 29,9 & 5,0622 & 5,1308\\
%\end{tabularx}
%\captionof{table}{Entfernungsmessung bei 10V Sendespannung}
%\label{tab:Entfernungsmessung10V}
%\end{minipage}\\
\chapter{Quellenangaben}
\subsection{Dokumentationen}

{ {[EUR] {Eurocircuits}}, {https://www.eurocircuits.com/pcb-design-guidelines/ }
\\ {22.\,März.\,2018}.\\

{ {[RNW] {RN-Wissen}}, {http://rn-wissen.de/wiki/index.php/Abblockkondensator }
\\ {22.\,März.\,2018}.\\

{ {[EV] {Elektronikpraxis Vogel}}, {https://www.elektronikpraxis.vogel.de/sieben-suenden-beim-\\leiterplatten-design-a-356703/ }
\\ {22.\,März.\,2018}.\\

{ {[Mik1] {Mikrocontroller.net}}, { https://www.mikrocontroller.net/articles/Richtiges\_Designen\_von\\
\_Platinenlayouts }
\\ {22.\,März.\,2018}.\\

{ {[Mik2] {Mikrocontroller.net}}, {https://www.mikrocontroller.net/articles/Schaltplan\_richtig\_zeichnen }
\\ {22.\,März.\,2018}.\\

{ {[INFI3] {Infineon}}, {https://www.infineon.com/cms/de/product/microcontroller/32-bit-industria\\
l-microcontroller-based-on-arm-cortex-m/32-bit-xmc1000-industrial-microcontroller-arm-cortex-m0/   }
\\ {22.\,März.\,2018}.\\

\subsection{Datenblätter}

{ {[NEX] {Nexperia}}, {https://assets.nexperia.com/documents/data-sheet/2N7002P.pdf}
\\ {22.\,März.\,2018}.\\

{{[INC1] {Doides Incorporated}}, {https://www.diodes.com/assets/Datasheets/ds30149.pdf}
\\{22.\,März.\,2018}.\\

{{[INC2] {Doides Incorporated}}, {https://www.diodes.com/assets/Datasheets/DMG6602SVT.pdf}
\\{22.\,März.\,2018}.\\

{{[INFI1] {Infineon}}, {https://www.infineon.com/dgdl/Infineon-XMC1400-DS-v01\_03-EN.pdf?\\fileId=5546d46250cc1fdf015110a2596343b2}
\\ {22.\,März.\,2018}.\\

{{{INFI2}}, {https://www.infineon.com/dgdl/Infineon-XMC1400-AA\_ReferenceManual-UM
\\-v01\_01-EN.pdf?fileId=5546d46250cc1fdf0150f6ebc29a7109}
\\ {22.\,März.\,2018}.\\

{{[KHM] {Elektrotechnik Karl-Heinz Mauz GmbH}}, {http://www.produktinfo.conrad.com/daten
\\blaetter/525000-549999/541348-da-01-en-OFFENER\_ULTRASCHALLSENSOR\_A\_16P\_T\_R.pdf}
\\{22.\,März.\,2018}.\\

{{[AM] { Allegro MicroSystems, LLC }}, {https://www.allegromicro.com/de-DE/Products/Motor-
\\Driver-And-Interface-ICs/Brush-DC-Motor-Drivers/A5950.aspx}
\\ {22.\,März.\,2018}.\\

{{[TI] {Texas Instruments}}, {http://pdf1.alldatasheet.com/datasheet-pdf/view/522033/TI/TLC272A.html}
\\{22.\,März.\,2018}.\\

\listoftables


\end{document}


