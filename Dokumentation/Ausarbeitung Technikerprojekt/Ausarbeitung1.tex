%Einfache Vorlage für eine mit Latex realisierte Hausarbeit von http://www.studieren-info.de
%Du kannst diese Vorlage für deine Hausarbeit beliebig anpassen%


%-------------------
%Beginn des Kopfbereiches
%-------------------

%Wir verwenden eine DIN-A4-Seite und die Schriftgröße 12.
\documentclass [11pt,a4paper,bibliography=totoc]{scrreprt}%Einfaches Dokument{article, scrartcl} komplexere Dokumente/Studienarbeiten {report, scrreprt} Bücher {book, scrbook} Präsentationen {beamer, seminar, texpower} Briefe {g-brief, scrlttr2}

%Diese drei Pakete benötigen wir für die Umlaute, Deutsche Silbentrennung etc.
%Apple-Nutzer sollten anstelle von \usepackage[latin1]{inputenc} das Paket \usepackage[applemac]{inputenc} verwenden
\usepackage[utf8]{inputenc}%[utf8] Umlaute direkt eingabe [latin1] für unixoide und windows
\usepackage[ngerman]{babel}
\usepackage[T1]{fontenc}
%Verwendung von vielen Farben
%\usepackage[usenames,dvipsnames]{color}
\usepackage{color}
% deutsches Literaturverzeichnis
\usepackage{bibgerm}
\usepackage{listings}%um Quellcode vernünftig einpflegen zu können
\lstset{language=C} 
\definecolor{codegreen}{rgb}{0,0.6,0}
\definecolor{codegray}{rgb}{0.5,0.5,0.5}
\definecolor{codepurple}{rgb}{0.58,0,0.82}
\definecolor{backcolour}{rgb}{0.95,0.95,0.92}
 
\lstdefinestyle{mystyle}{
    backgroundcolor=\color{backcolour},   
    commentstyle=\color{codegreen},
    keywordstyle=\color{magenta},
    numberstyle=\tiny\color{codegray},
    stringstyle=\color{codepurple},
    basicstyle=\footnotesize,
    breakatwhitespace=false,         
    breaklines=true,                 
    captionpos=b,                    
    keepspaces=true,                 
    numbers=left,                    
    numbersep=5pt,                  
    showspaces=false,                
    showstringspaces=false,
    showtabs=false,                  
    tabsize=2
}
 
\lstset{style=mystyle}


%nutzung der verlinkten formeln
\usepackage{amsmath}
%Das Paket erzeugt ein anklickbares Verzeichnis in der PDF-Datei.
\usepackage{hyperref}
%Das Paket wird für die anderthalb-zeiligen Zeilenabstand benötigt
\usepackage{setspace}

%erweitrte mathematische Tabellenformatierung
\usepackage{array}
% Grafiken verwenden
\usepackage{graphicx}
\usepackage{float}
\usepackage{struktex}
%Funktion zum skallieren
\makeatletter
\def\ScaleIfNeeded{%
\ifdim\Gin@nat@width>\linewidth
\linewidth
\else
\Gin@nat@width
\fi
}
\makeatother

%\tiny{winzig}
%\small{klein}
%\large{groß}
%\Large{bisschen größer}
%\huge{riesig}
%\Huge{Riesig}

%\textbf{Fett}
%\textit{Kursiv}
%\emph{Kursiv}	
%\textsl{schief}
%\textsc{Kapit\"alchen}
%\textsf{Sans Serif}
%\textrm{Roman}
%\texttt{Schreibmaschine}
%\textnormal{Normale Schrift}
%\underline{unterstrichen}
%\footnote{XXX}
%\includegraphics[width=80pt]{Unterschrift.jpg}

%\textcolor{white}{bla...}
%{GreenYellow, Yellow, Goldenrod, Dandelion, Apricot, Peach, Melon, YellowOrange, Orange, BurntOrange, Bittersweet, RedOrange, Mahogany, Maroon, BrickRed, Red, OrangeRed, RubineRed, WildStrawberry, Salmon, CarnationPink, Magenta, VioletRed, Rhodamine, Mulberry, RedViolet, Fuchsia, Lavender, Thistle, Orchid, DarkOrchid, Purple, Plum, Violet, RoyalPurple, BlueViolet, Periwinkle, CadetBlue, CornflowerBlue, MidnightBlue, NavyBlue, RoyalBlue, blue, Blue, Cerulean, Cyan, ProcessBlue, SkyBlue, Turquoise, TealBlue, Aquamarine, BlueGreen, Emerald, JungleGreen, SeaGreen, Green, ForestGreen, PineGreen, LimeGreen, YellowGreen, SpringGreen, OliveGreen, RawSienna, Sepia, Brown, Tan, Gray}

%Einrückung eines neuen Absatzes
\setlength{\parindent}{0em}
%Definition der Ränder
\usepackage[paper=a4paper,left=2.5cm, right=2.0cm, top=2.5cm, bottom=2.0cm]{geometry}
%anpassung der Farben der Überschriften
\addtokomafont{disposition}{\color{black}}
%Abstand der Fußnoten
\deffootnote{1em}{1em}{\textsuperscript{\thefootnotemark\ }}

\addtolength{\footskip}{-1cm}% Fußbereich 1 cm höher setzen
%section size 
%\usepackage{titlesec}
%\titleformat{\chapter}[display]
%{\normalfont%
%    \huge% %change this size to your needs for the first line
%    \bfseries}{\chaptertitlename\ \thechapter}{15pt}{%
%    \Huge %change this size to your needs for the second line
%    }
%%\titleformat*{\chapter}{\fontsize{15}{20}\selectfont}
%\titleformat*{\section}{\fontsize{13}{20}\selectfont}
%\titleformat*{\subsection}{\fontsize{12}{17}\selectfont}

%Regeln, bis zu welcher Tiefe (section,subsection,subsubsection) überschriften angezeigt werden sollen (Anzeige der überschriften im Verzeichnis / Anzeige der Nummerierung)
\setcounter{tocdepth}{3}
\setcounter{secnumdepth}{3}

%-------------------
%Ende des Kopfbereiches
%-------------------



%-------------------
%Hier beginnt der Text deiner Hausarbeit
%-------------------
\begin{document}


%Beginn der Titelseite
\thispagestyle{empty}
\begin{center}
\begin{Huge}
\textcolor{blue}{\textbf{Entwicklung und testen eines Ultraschall-Entfernungsmessers als Vorbereitung eines Produktentwurfes}}
\end{Huge}
\rule{\textwidth}{.4pt}
\vspace{1.5cm}
% Weiter in großer Schrift

\huge{\textbf{Projektarbeit}}\\
\begin{Large}
erstellt an der\\
Fachschule für Technik des Carl-Severing-Berufskolleg\\
für Metall- und Elektrotechnik der Stadt Bielefeld\\
\includegraphics[width=100pt]{Abbildungen/CSBlogo.png}\\

Erstellt durch:\\
\vspace{12pt}
Eduard Meiser\\Omar Hachimi \\Stephan Dannat\\FET6A\\
\vspace{12pt}
in Zusammenarbeit mit der Fa. Tinkerforge\\
betreut durch\\
Herr Simon\\
Bielefeld, \today
\end{Large}
\end{center}
\newpage
\thispagestyle{empty}
\textbf{\Large{Persönliche Erklärung}}\vspace{10pt}

Hiermit bestätigen wir, dass die vorliegende Arbeit selbstständig verfasst und keine anderen als die angegebenen Hilfsmittel benutzt wurden. Die Stellen der Arbeit, die dem Wortlaut oder dem Sinn nach anderen Werken (dazu zählen auch Internet-quellen) entnommen sind, wurden unter Angabe der Quellen kenntlich gemacht.

\vspace{50pt}


\noindent Bielefeld,\noindent\rule{4cm}{.4pt} \hfill\rule{5cm}{.4pt}\par
\hfill Eduard Meiser 

\vspace{30pt}
%\noindent\rule{5cm}{.4pt}
\hfill\rule{5cm}{.4pt}\par
%\noindent Ort, Datum
\hfill Omar Hachimi 

\vspace{30pt}
%\noindent\rule{5cm}{.4pt}
\hfill\rule{5cm}{.4pt}\par
%\noindent Ort, Datum 
\hfill Stephan Dannat 
%				Inhaltsverzeichnis
\setcounter{page}{0}
\newpage 
\pagenumbering{roman} 
\tableofcontents 
\newpage
%				Start der eigentlichen Arbeit
\newpage
\setcounter{page}{0}
\pagenumbering{arabic}

\chapter{Lasten und Pflichtenheft}
\section{Lastenheft}
\subsection{Über Tinkerforge}
Die  Tinkerforge  GmbH  wurde  Ende  2011  mit  dem  Ziel  gegründet,  die  Handhabung eingebetteter  Systeme  zu  vereinfachen.  Das  Tinkerforge  Baukastensystem  besteht  aus aktuell fast 80 verschiedenen Modulen, die vom Anwender flexibel für die jeweilige Aufgabe  kombiniert  werden  können.  Zu  den  Modulen  zählen  diverse  Sensor-  Aktor-  und Schnittstellenmodule, die alle über Hochsprachen wie C\#, Python und Java gesteuert werden können. Tinkerforge unterstützt aktuell 17 verschiedene Programmiersprachen. Sowohl Hardware als auch die Software aller Module sind OpenSource. Die Stärke des Baukastensystems  ist  aus  Anwendersicht  die  enorme  Flexibilität,  die  Einfachheit  und
die Schnelligkeit mit der Projekte realisiert werden können. Es eignet sich daher besonders im Bereich Rapid Prototyping. Daher findet das Tinkerforge Baukastensystem Anwendung in vielen Forschungsinstituten, in diversen Entwicklungsabteilungen bekannter Automobilhersteller und Ingenieurbüros.
\subsection{Motivation}
Diese Technikerarbeit soll die Grundlage zur Entwicklung eines Entfernungssensors für das Baukastensystem bilden, der auf einer Ultraschall-Entfernungsmessung basiert. Das Baukastensystem verfügt aktuell über so einen Sensor. Bei diesem handelt es sich aber im wesentlichen um ein zugekauftes Modul, welches nicht die gewünschten Leistungen liefert. Daher soll an einem zu entwerfenden Prototypen Forschung betrieben werden, um eine eigene Lösung entwerfen zu können.
\subsection{Aufgabenbeschreibung}
Innerhalb dieser Arbeit soll der Entwurf eines Prototypen des Entfernungssensors und die damit verbundene Forschung durchgeführt werden. Dabei ist durch Recherche zu erarbeiten, welche Möglichkeiten zur Realisierung zur Verfügung stehen. Durch Messungen am Prototypen soll festgestellt werden, welche dieser Möglichkeiten funktional und finanziell realisierbar sind, um ein eigenes Produkt zu erstellen. Sollte im Anschluss noch die Möglichkeit bestehen, sind die Ergebnisse in ein serienreifes Modul umzusetzen.
\newline\newline
Diverse Teilaufgaben sind zu erledigen: \newline
\begin{itemize}
\item \textbf{Recherche}\newline
Zu  Anfang  muss  recherchiert  werden,  welche  Möglichkeiten  es  gibt  mittels  Ultraschall eine Entfernung zu ermitteln und wie diese technisch umgesetzt werden können. Zusätzlich müssen die Techniker sich mit dem Tinkerforge Baukastensystem und seiner internen Funktionsweise vertraut machen.
\item \textbf{Bauteilauswahl}\newline
Abhängig  von  der  gewählten  technischen  Umsetzung  müssen  geeignete  Komponenten ausgewählt werden. Die Auswahl sollte auch unter dem Gesichtspunkten Preis, der Bauteilverfügbarkeit und der technischen Anforderungen erfolgen.\\
\item \textbf{Schaltplanentwurf und Layouterstellung}\newline
Von  Tinkerforge  wird  das  Open  Source  CAD  Programm  KiCad  verwendet.  Mit diesem Programm ist ein Schaltplan für den Prototypen und anschließend ein Leiterplattenlayout zu erstellen.
\item \textbf{Leiterplattenbestückung}\\
Die erstellte Leiterplatte wird von Tinkerforge in Auftrag gegeben. Diese muss mit den gewählten Komponenten bestückt werden. Die Tinkerforge GmbH stellt dazu die notwendigen Werkzeuge bereit.
\item \textbf{Einrichten und Einarbeitung in die Tinkerforge Toolchain}\\
Viele  Softwarekomponenten  werden  von  der  Tinkerforge  Toolchain  automatisch generiert. Um diese Nutzen zu können muss ein Linux System %in einer virtuellen Maschine
eingerichtet werden. Anschließend muss sich mit der Funktionsweise des Generators und der Softwareversionsverwaltung"Git" vertraut gemacht werden.
\item \textbf{Testsoftware und Forschung}\\
Um Messungen an der Hardware durchführen zu können gilt es Programmblöcke zu entwerfen, mit denen die einzelnen Funktionen der Baugruppen getestet werden können. So soll ermittelt werden, wie zum einen das Ultraschallsignal effektiv ausgegeben werden kann und wie sich die Signalamplitude auf die Reichweite auswirkt. Zum anderen gilt es zu recherchieren, wie das zurückkommende Signal verarbeitet werden kann. Auch soll erarbeitet werden, wie gut das Signal unter verschiedenen Bedingungen verarbeitet werden kann und ob eine zuverlässige Verarbeitungsqualität ohne großen Aufwand realisierbar ist.
\end{itemize}
\chapter{Management des Projektes}

\subsection{Trello}
Zur zeitlichen Planung und Übersicht des Ablaufes wurde auf das Onlinetool Trello zurückgegriffen. Dieses ist ein kostenfreies, webbasiertes Projektmanagementtool. Es ermöglicht den Gruppenmitgliedern gleichzeitig von verschiedenen Orten auf die Oberfläche zuzugreifen und Änderungen vorzunehmen. So kann ein Teilnehmer auch neue Termine mit Kennzeichnung der Fälligkeit für andere Gruppenmitglieder einfügen, oder bereits erledigte Aufgaben für alle abhaken. Auch können hier relevante Dokumente, die alle Gruppenmitglieder lesen sollen hochgeladen, und bei Bedarf noch kommentiert werden. Für die Dokumentation lässt sich an diesem System auch einfach abgleichen, zu welchen Zeitpunkten die einzelnen Aufgaben abgeschlossen wurden.

\subsection{Github}
Bei Github handelt es sich um einen webbasierten Online-Dienst, der Server für Entwicklungsprojekte mit einer Versionsverwaltung bereitstellt. So können alle Daten nach einer Änderung im Programm wieder hochgeladen und mit einem Kommentar versehen werden. Sollte nach mehreren Veränderungen ein Problem auftreten, kann auf eine ältere Version zurückgegriffen und somit kann der Fehler eingegrenzt werden. Auch kann ein Projekt in Teilabschnitte aufgeteilt werden, damit mehrere Personen unabhängig voneinander daran arbeiten können. Nach der Bearbeitung können diese wieder zusammengefügt werden. Dabei ist erkennbar, welche Änderungen, von wem vorgenommen wurden. So können alle Vorgänge jederzeit verfolgt werden, um eine größtmögliche Übersicht zu gewährleisten. Durch das Kommentieren der Änderungen kann die Nachvollziehbarkeit dieser ebenfalls deutlich gesteigert werden. Des weiteren ist diese Plattform gerade für Unternehmen wie Tinkerforge, die ihren Quellcode als Open-Source anbieten besonders praktisch, da den Nutzern hier alle veröffentlichten Daten direkt zur Verfügung stehen.
\chapter{Recherche der Funktionsweise}
\section{Sender}
Für den Sender (und empfänger) wurde ein auf Pietzomodulen basierendes Kapselgehäuse verwendet. Dabei wurden für den Prototypen mehrere Sender vereschiedener Hersteller bestellt, um Unterschiede der verschiedenpreisigen Bauteile zu ermitteln und festzustellen, welches Preissegment die nötige Qualität für die vorliegende Anwendung erfüllt.

\section{H-Brücke}
Der Lautsprecher wird mit einer Frequenz von 40kHz und einer Amplitude von 20V P-P betrieben. Um ein positives sowie negatives Rechtecksignal zu generieren wird eine H-Brücke verwendet. Sie Abb.x Funktion Block Diagramm vom IC A5950\\
In der Abb. Funktion Block Diagramm vom IC A5950 wird an OUT 1 sowie an OUT 2 eine Stromumkehrung erzielt, durch die Ansteuerung von den MOS-FET von dem Control Logic. Eine genaue Funktionsanalyse ist nicht erforderlich, weil an der H-Brücke nur ein Ultraschallsensor
angeschlossen wird. Wichtig für die H-Brücke ist das sie mit 40kHz schalten kann die am Eingang. Ausgründen der internen Beschaltung und Toleranzen sind am Ausgang nicht die vollen 40kHz zu erwarten, sondern ein Verzug was später im Kapitel X.X behandelt wird.
\section{Empfänger}
Abb.1\\
In der obigen Schaltung(Bildverweis) ist zu sehen dass das ankommende Sinusförmige Signal, verstärkt und in ein digitales Signal umgewandelt wird. Die Schaltung wurde mit einem Hochpassfilter (CR Glied) bestückt bestehend aus C12 und R5 um unerwünschte Signalanteile von unter 40kHz zu unterdrücken. Der Widerstand wurde nach der e24 Reihe, anhand der folgende Berechnung, ausgewählt:
\onehalfspacing \\
\(\displaystyle C=\frac{1}{2*pi*fg*R}\Rightarrow\frac{1}{2*pi*40kHz*C12*100 Ohm}\approx40pF \)
\singlespacing
Die Kapazität des Kondensators C12 wurde an die Grenzfrequenz von 40 kHz und den Widerstand angepasst.\\
Für die Verstärkung der Amplitude so wie der Umwandlung des analog Signals in ein Rechtecksignal mit 40 KHz  dienen die Operationsverstärker des Bausteines TLC272. Die Versorgungsspannung der OPV's von 3,3V wird durch den Kondensator C16 (EMV Störfilter) stabilisiert.\\
Für die Verstärkung der Amplitude ist der Operationsverstärker TLC272 U2B als  nicht invertierender Verstärker geschaltet.
Wenn eine Gleichspannung anliegt, wirkt der Kondensator (C10) in der Operationsverstärkerschaltung als Impedanzwandler, also mit einer Verstärkung von eins, geht nun die Eingangsfrequenz hoch, nimmt der widerstand des Kondensators (C10) ab, somit beginnt der Operationsverstärker auch zu verstärken, und zwar mit zunehmender Verstärkung, bis irgendwann die Impedanz des Kondensators vernachlässigt werden kann und die Verstärkung nur noch durch das Verhältnis der Widerstände beeinflusst wird, R6 ist zudem notwendig um das schwingen der Amplitude zu verhindern, somit kann die Verstärkung mit folgender Formel berechnet werden:
\onehalfspacing \\
\(\displaystyle Vu=R6+R8+\frac{R12}{R6} .\) 
\singlespacing
Die Z-Diode D2 ist für die Spannungsstabilisierung und als Sicherung da.\\
Für die Umwandlung des Analogen Signales in ein Digitales wurde der Operationsverstärker TLC272 U2C als Komparator geschaltet. Beim auftreten von Differenzen zwischen den eingangs Signalen, wechselt der Ausgang des Komparators zwischen Low (0 Volt) auf High (3,3 Volt).

\section{Hochsetzsteller}
Der Hochsetzsteller dient dazu, aus den 5V Versorgungsspannung eine (für die Versuche variable) höhere Spannung für den Sendebetrieb zu schaffen. So kann der Schalldruck der ausgegeben wird erhöht werden(größere Reichweite/größeres Rücksignal)\\
Die Funktionsweise des Hochsetzstellers (Spannungspumpe/Aufwärtswandler/Aufwärtsregler) ist relativ simpel und findet in vielen Bereichen Anwendung. Grundsätzlich wird eine Induktivität in Reihe mit einer Freilaufdiode vor einen Ladecondensator geschaltet. Dieser liegt parallel zum Ausgang. Zwischen der Spule und der Diode ist ein Schalter angeschlossen, der die Spule gegen Masse schaltet. So läd sich die Spule bei Betätigung des Schalters auf (durch den Stromfluss entsteht ein Magnetfeld) und beim Öffnen steigt die Spannung am sekundären Ende der Spule, durch das zusammenbrechende Magnetfeld, an und läd den Kondensator auf. Dieser Vorgang wird wiederholt, bis der Kondensator so weit aufgeladen ist, dass die Ausgangsspannung den gewünschten Wert hält. Dann wird die Schaltfrequenz auf das mindestmaß verringert, um den Wert zu halten. Natürlich ist die mögliche Ausgangsspannung nicht unbegrenzt über das Schaltspiel regelbar, sondern ist auch von den Baugrößen der Bauteile abhängig. Mit einer Induktivität von 10mH und einem Kapazität von 40uF lässt sich die Ausgangsspannung bei 5v Eingangsspannung zwischen 6v und 20v einstellen.


\section{Mikroprozessor}
Der Infineon XMC 1xxx48 gehört zu der Familie der ARM Cortex -M0 Prozessoren und ist ein 32-bit Industrial Microcontroller und wird mit 48MHz externer Clock betrieben. Die 48 steht für die Anzahl der Pins. Der interne Timer läuft mit 96Mhz. Neben bietet der XMC einen 12 bit A/D Wandler, welcher für die Analogmessung eine viel genauere Auflösung bieten kann als ein 8 bit A/D Wandler. Die Betriebsspannung des Prozessors beträgt 3,3V.\\
https://www.infineon.com/cms/en/product/microcontroller/32-bit- industrial-microcontroller- based-
on-arm- cortex-m/32- bit-xmc1000- industrial-microcontroller- arm-cortex- m0/\#\\

\chapter{Wahl sowie alternative Bauteile}
Für den Prototypen wurden mehrere verschiedene Ultraschallkapseln verwendet, um einen Vergleich der Qualität der Bauteile herstellen zu können. Für die Operationsverstärker wurde das IC LS232 verwendet, da dieses IC bereits zwei OPV's enthält.
\chapter{Open Source CAD-Programm Packet zur Erstellung von elektronischen Leiterplatten}
\input{5_Open_Source_CAD-Programm}
\chapter{Entwicklung der Software zum Betrieb des Prototypen}
\input{6_Entwicklung_der_Software}
\chapter{Messungen und Auswertung der Ergebnisse}

Zu Beginn wurden Messungen direkt an dem Prozessor durchgeführt, um zu kontrollieren, dass die im Programm errechneten Signale auch die vorgesehenen Frequenzen erreichen. Anschließend wurden einzelne Baugruppen mit dem Prozessor kombiniert, um das Zusammenspiel dieser auswerten zu können. Danach wurden die Baugruppen um den Sensor erweitert und auch dieses Zusammenspiel wurde ausgewertet. Zum Abschluss wurde dann die komplette Platine im Betrieb getestet und optimiert.

\section{PWM-Ausgabe}
Um das Signal für die Entfernungsmessung zu generieren wurde der Mikrocontroller so programmiert, dass zehn Impulse mit einer Frequenz von 40kHz ausgegeben werden. Danach erfolgt eine Pause, um das zurückkehrende Signal abzuwarten und auszuwerten.\\
\begin{figure}[H]
\includegraphics[width=1.0\textwidth]{Abbildungen/PWM-Signal.png}\caption{PWM-Burst auf 40kHz Basis}\label{fig:pwm-burst}
\end{figure}
In der Abbildung \ref{fig:pwm-burst} ist zu sehen, dass ein Burst aus zehn Impulsen mit einer Periodendauer von jeweils 25us generiert wurde. Diese Messung wurde direkt an dem Mikrocontroller vorgenommen, um sicherzustellen dass die H-Brücke mit der passenden Frequenz angesteuert wird.
\begin{figure}[H]
\includegraphics[width=1.0\textwidth]{Abbildungen/Abstand1.png}\caption{Versuchsmessung mit Sender und Empfänger parallel}\label{fig:Abstand1}
\end{figure}
In der Abbildung \ref{Abstand1} ist zu sehen, wie das Eingangssignal am Empfänger aussieht, wenn der Sender und der Empfänger getrennt aufgebaut sind und parallel auf ein Hinderniss gerichtet sind. Dabei zeigt die graue Linie den Verlauf des Sendersignals und die schwarze Linie den Verlauf des Empfängersignals
\section{H-Brücke}
Die H-Brücke wurde durch einen IXDN602, einen MOSFET ersetzt. Dieser wurde als Low-Side Driver-geschaltet. Dies geschah, weil erstens die Beschaltung des OPV nur für eine positive Halbwelle funktioniert und zweitens die Beschaltung der H-Brücke fehlerhaft war. So wurde einer der Ausgänge auf Masse gelegt, was im Betrieb einen Kurzschluss verursachen würde. Um diese Probleme zu beheben, müssten der Sender- und der Empfängerkreis aufgetrennt werden und eine zweite Ultrashallkapsel eingesetzt werden.

\section{Signalverlauf nach Verstärkung durch den OPV}

\section{aufgetretene Probleme}
Zur Inbetriebnahme der H-Brücke mussten Korrekturen vorgenommen werden. Denn diese gab zu Anfang noch keine Signale aus. Es wurde übersehen, dass der OCLSEL-Pin (Oszillatorauswahl) bei der Beschaltung berücksichtigt werden muss. So wurde der Prototyp um eine Drahtbrücke erweitert.

\chapter{Ergebnisse der Forschung}
Die Verwendung der H-Brücke war eine Fehlentscheidung, diese wurde durch einen MOSFET ersetzt. Das Problem mit der H-Brücke bestand darin, dass das gewählte Bauteil zum einen nicht zwangsläufig die gewünschte Schaltfrequenz halten kann.

\chapter{Fazit aus den Ergebnissen für denAuftraggeber}
\input{9_Fazit_Auftraggeber}
\chapter{Reflektion über den Projektablauf}
\input{10_Reflektion_Ablauf}
\chapter{Anhänge}
\section{Schaltpläne}

\section{Platinenlayout}

\section{Quellcode}
Hier könnte Ihr Quellcode (inklusive Werbung) stehen\\
%\input{12_Quellcode}


weitere Abbildungen
\listoffigures
\listoftables


\end{document}


