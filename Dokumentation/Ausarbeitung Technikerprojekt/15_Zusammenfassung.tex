\section{Zusammenfassung}
Das Ziel des Technikerprojekts war das Entwickeln und testen eines Ultraschall-Entfernungsmessers als Vorbereitung eines Produktentwurfes. Dabei gab es zwei Möglichkeiten den Ultraschall- Entfernungsmesser zu realisieren. Begonnen wurde mit einer Variante mit getrenntem Sende- und Empfangsbetrieb, auf zwei Leiterplatten aufgebaut. Bei den Tests dieser wurden Fehler behoben und Verbesserungen integriert. Danach entstand eine neue Variante mit kombiniertem Betrieb, auf einer Platine aufgebaut. Diese zweite Prototyp-Version besaß nur noch eine Ultraschallkapsel für den Sende- und Empfangsbetrieb. Mit dieser zweiten Prototyp-Version wurden zahlreiche Messungen durchgeführt um aufzuzeigen, dass die Ergebnisse mit denen einer zwei-Kapsel Variante vergleichbar sind.
Aus den Messungen und Tests ist klar ersichtlich, dass ein Ultraschall-Entfernungsmesser mit nur einer Ultraschallkapsel realisierbar ist. Die Messgenauigkeiten liegen im Bereich von wenigen Zentimetern. Diese lassen sich durch Anpassungen in der Software noch weiter verbessern. Das Ergebnis ist ein funktionaler Prototyp.


