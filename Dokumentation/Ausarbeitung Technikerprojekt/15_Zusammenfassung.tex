\section{Abstract}
Das Ziel des Technikerprojekt war das Entwickeln und testen eines Ultraschall-Entfernungsmessers als Vorbereitung eines Produktentwurfs. Das Ergebnis ist ein vorläufig funktionaler Prototyp deren Software noch Optimiert werden muss.
Aus den Messungen und Tests ist klar ersichtlich, dass es mit einer Kapsel funktioniert, die sowohl ein Ultraschallsignal sendet als auch empfängt. Hierbei hat sich die Ultraschallkapsel des Herstellers EKULT als die beste Kapsel herauskistalisiert, da sie weder vertauscht noch verpolt werden kann und sich somit für das Senden und Empfangen eignet.
Die Messgenauigkeiten liegen im Bereich von 4 bis 4,5\,cm. Durch einen Korrekturwert in der Software könnte die Genauigkeit auf einen Zentimeter verbessert werden.
Die Experementierumgebungen waren unterschiedliche Entfernungsmessungen mit verschiedenen Kapseln, sowie mit variabler Amplitude. Die Resultate der Forschungsarbeit sind für Tinkerforge interessant, weil sie die Grundlage für lohnende Weiterentwicklungen darlegt.



