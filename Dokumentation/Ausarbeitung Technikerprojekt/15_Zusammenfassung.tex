\section{Abstract}
Das Ziel des Technikerprojekt war es, das Entwickeln und Testen eines Ultraschall-Entfernungsmessers als Vorbereitung eines Produktentwurfes. Das Ergebnis ist ein vorläufig funktionaler Prototyp deren Software noch Optimiert werden muss.
Aus den Messungen und Tests ist klar ersichtlich, dass es mit einer Kapsel funktioniert die sowohl ein Ultraschallsignal sendet und als auch empfängt. Hierbei hat sich die Ultraschallkaplsen des Hersteller EKULT herauskistalisiert als die bessere Kapsell weder vertauscht noch verpolt werden kann und somit sich für das Senden und Empfangen geeignet ist.
Die Messgenauigkeiten liegen alle im Bereich von 4\,cm bis 4,5\,cm, somit könnte ein Korrekturwert in die Software eingearbeitet werden um auf ein Minimum von einem Zentimeter die genauigkeit zu erzielen.
Die Experementierumsich gebungen waren unterschiedliche Entfernungsmessungen mit verschiedenen Kapseln, sowie mit variabler Amplitude. Die Resultate der Forschungsarbeit sind für Tinkerforge interessant, weil sie die Grundlage für lohnende Weiterentwicklungen darlegt.



