Für die Messungen wurden Laufe des Projekts zwei Versionen an Prototyp-Platinen entworfen, an denen Messungen und Verbesserungen vorgenommen wurden. 
\section{Prototyp 1}
Bei dem ersten Prototyp wurden die Sendereinheit und die Empfängereinheit auf getrennten Platinen aufgebaut. So bestand die Möglichkeit, den Senderkreis und den Empfängerkreis getrennt zu untersuchen, ohne dass sich elektrische Signale der beiden Schaltkreise überlagern konnten.

\subsection{Senderkreis}
Zu erst wurden Signale direkt an der CPU gemessen, um sicher zu stellen, dass die Einstellungen im Programm auch die gewünschten Ausgaben zur Folge haben, und keine Gefärdung der Bauteile entsteht.
Um das Signal für die Entfernungsmessung zu generieren wurde der Mikrocontroller so programmiert, dass zehn Impulse mit einer Frequenz von 40kHz ausgegeben werden. Danach erfolgt eine Pause, um das zurückkehrende Signal abzuwarten und auszuwerten.\\
\begin{minipage}{0.5\textwidth}
\includegraphics[width=1\textwidth%, draft
]{Abbildungen/MessungenP1/PWM-von-der-cpu.png}
\captionof{figure}{PWM-Burst auf 40kHz Basis an der CPU}
\label{fig:pwm-burst}
\end{minipage}
\begin{minipage}{0.5\textwidth}
\includegraphics[width=1\textwidth%, draft
]{Abbildungen/MessungenP1/PWM-ausgabe-mit-Hi-Side.png}
\captionof{figure}{PWM Ausgabe über einen Hi-Side}
\label{fig:HiSide}
\end{minipage}
In der Abbildung \ref{fig:pwm-burst} ist zu sehen, dass der gewünschte Burst aus zehn Impulsen mit einer Periodendauer von jeweils 25us vom Mikrocontroller generiert wurde. Diese Messung wurde auch vorgenommen, um zu überprüfen, wie sich das Signal durch die eingesetzten Bauteile verändert.\\
Die Abbildung \ref{fig:HiSide} zeigt, wie das Ausgangssignal nach einer Hi-Side aussieht. So wird zwar im Takt des PWM-Signals geschaltet, allerdings fehlt es an einem Gegenpool, um das Potential in den Schaltpausen wieder auf Null zu ziehen. Dadurch bleibt die Spannung während des Schaltens immer auf einem erhöhten Pegel und sinkt erst nach Ende des PWM-Signals langsam ab. Dadurch kann natürlich keine vernünftige Ausgabe am Lautsprecher erzeugt werden, denn ohne deutliche Potentialunterschiede kann dieser auch nicht in Schwingungen versetzt werden. Der ausgegebene Schalldruck würde maximal für kürzeste Entfernungsmessungen reichen, wenn überhaupt und dann würde das zurückkommende Signal noch von der abklingenden Spannung des Hi-Side überlagert. Somit ist dieser Aufbau nicht operabel.\\
Um die Spannung nicht nur auf einen Hi-Pegel, sondern auch auf einen LOW-Pegel schalten zu können wurde danach auf eine Halbbrücke gewechselt. Mit dieser lässt sich der Ausgang, über zwei durch das PWM-Signal gesteuerte MOSFETs, sauber auf Hi- oder LOW-Pegel schalten. %Bei der ersten, einfach gesteuerten Version, entstand das Problem, dass beim Schalten der Halbbrücke, beide MOSFETs gleichzeitig geschaltet haben. Dieses klingt zwar nicht nach einem Problem, doch wird es durch die technischen Gegebenheiten zu einem, denn bei den meisten elektronischen Schaltern verläuft der Einschaltprozess deutlich schneller, als der Ausschaltprozess. Wenn also zwei Bauteile gleichzeitig schalten, hat das einschaltende Bauteil schneller eingeschaltet, als das ausschaltende Bauteil ausgeschaltet. Dadurch entstehen bei jedem Schaltvorgang Kurzschlüsse. Auch wenn diese nur für Nanosekunden bestehen, bevor sie wieder unterbrochen werden, ist davon auszugehen, dass zu den dadurch entstehenden Störungen auch Bauteile zerstört werden.\\ Somit wurde auf eine voll gesteuerte Halbbrücke gewechselt und es wurden zwei PWM-Signale moduliert, bei denen eines invertiert und die Flanken derart verschoben waren, dass die angesteuerten MOSFETs nie Kurzschlüsse schalten können.
Mit der verwendeten Halbbrücke ergab sich die Abbildung \ref{fig:Halfbridge}\\
\begin{minipage}{0.5\textwidth}
\includegraphics[width=1\textwidth%, draft
]{Abbildungen/MessungenP1/PWM-Nach-der-Halbbrucke.png}
\captionof{figure}{PWM Ausgabe über eine Halbbrucke}
\label{fig:Halfbridge}
\end{minipage}
\begin{minipage}{0.5\textwidth}
\includegraphics[width=1\textwidth%, draft
]{Abbildungen/MessungenP1/PWM-Nach-der-Halbbrucke-mit-LS.png}
\captionof{figure}{Ausgabe der PWM an der Ultraschallkapsel}
\label{fig:Senderausgabe}
\end{minipage}
Es zeigt sich, dass das Signal nach der Erweiterung auf eine Halbbrücke wieder wie das von der CPU ausgegebene PWM-Signal \ref{fig:pwm-burst} aussieht, nur dass die Amplitude wie geplant höher ausfällt. Somit kann die Höhe der Amplitude über die Spannungspumpe variiert werden um die Stärke des ausgegebenen Signals zu verändern, ohne die CPU durch die höhere Spannung zu beschädigen. Wie in der Abbildung \ref{fig:Senderausgabe} zu entnehmen ist, entstehen durch die angeschlossene Ultraschallkapsel höhere Spannungsimpulse im Einschaltmoment.

\subsection{Empfängerkreis}
Die Platinen des Sender- und Empfängerkreises wurden gemeinsam auf einer Halterung montiert, so dass die Ultraschallkapseln zum senden und empfangen der Signale nebeneinander befestigt werden konnten. Ziel war es, durch verschieben eines Hindernisses die Signaländerungen an den Platinen beobachten zu können, ohne die gesamten Messaufbauten bewegen zu müssen. \\
\begin{minipage}{0.5\textwidth}
\includegraphics[width=1\textwidth%, draft
]{Abbildungen/MessungenP1/Signal-Empfang.png}
\captionof{figure}{Signal Empfang}
\label{fig:Empfang am LS}
\end{minipage}
\begin{minipage}{0.5\textwidth}
\includegraphics[width=1\textwidth%, draft
]{Abbildungen/MessungenP1/Signal-nach-Verstarkung.png}
\captionof{figure}{Signal nach Verstärkung}
\label{fig:Verstaerkung}
\end{minipage}
%\begin{minipage}{0.5\textwidth}
%\includegraphics[width=1\textwidth%, draft
%]{Abbildungen/MessungenP1/Signal-nach-der-Filterung.png}
%\captionof{figure}{Signal nach der Filterung}
%\label{fig:Filterung}
%\end{minipage}
Die Abbildung \ref{fig:Empfang am LS} zeigt das Signal, das direkt am Empfänger zu messen war. Hier sind verschiedene vorerst nicht zuordenbare Signale zu sehen. Allein aus diesem Bild lässt sich aber keine Aussage zu den Signalen machen. Fest steht nur, dass ebenfalls Signale die nicht der gewünschten Frequenz entsprechen, vom Empfänger aufgenommen werden. Dies gilt es natürlich schnellst möglich auszumerzen, um unerwünschte Störungen zu vermeiden.
Die Abbildungen \ref{fig:Verstaerkung} und \ref{fig:Verstaerkung2} zeigen den Verlauf des Signals nach der Filterung und Verstärkung in zwei verschiedenen Zeitauflösungen. Dabei entspricht \ref{fig:Verstaerkung} den ersten drei Kästchen von \ref{fig:Verstaerkung2} und dient um darzustellen, dass die Verstärkung eine maximale Aussteuerung von 3,3V nicht überschreitet.
%In der Abbildung \ref{fig:Filterung} ist das Signal nach dem eingebauten Hochpassfilter zu sehen.\\

\begin{minipage}{0.5\textwidth}
\includegraphics[width=1\textwidth%, draft
]{Abbildungen/MessungenP1/Signal-nach-Verstarkung2.png}
\captionof{figure}{Signal nach Verstärkung2}
\label{fig:Verstaerkung2}
\end{minipage}
\begin{minipage}{0.5\textwidth}
\includegraphics[width=1\textwidth%, draft
]{Abbildungen/MessungenP1/Signal-nach-Komparator.png}
\captionof{figure}{Signal nach Komparator}
\label{fig:Komparator}
\end{minipage}\\
Nach dem das Signal den Komparator passiert hat, ergibt sich das Bild wie in Abbildung \ref{fig:Komparator} zu sehen ist. Bei einem Vergleich mit dem Signal nach der Verstärkung \ref{fig:Verstaerkung2} wird sichtbar, dass der Komparator nur Signale, die über seinem Schwellwert liegen, durchschaltet. Die Aufteilung in zwei Signalblöcke in den Abbildungen kommt daher, dass der erste Block das Signal der Sender-Kapsel ist, das direkt beim Senden seitlich auf die Empfänger-Kapsel abgestrahlt wurde. Der zweite Block ist bereits das Echo, das vom 20cm entfernten Hindernis zurückgeworfen wurde.\\
%Nachfolgend wurden die Signalverläufe mit verschiedenen Ultraschallkapseln aufgenommen, um vergleichen zu können, wie die Signalqualität bei verschiedenen Produkten schwankt. Dabei wurde die Amplitude von 4,6V für das Sendersignal und die Verstärkung des Empfängers auf ein Minimum eingestellt.\\
%\begin{minipage}{0.5\textwidth}
%\includegraphics[width=1\textwidth%, draft
%]{Abbildungen/MessungenP1/EKULIT1,5m.png}
%\captionof{figure}{EKULIT 1,5m}
%\label{fig:EKULIT1,5m}
%\end{minipage}
%\begin{minipage}{0.5\textwidth}
%\includegraphics[width=1\textwidth%, draft
%]{Abbildungen/MessungenP1/MURATAr1,5m.png}
%\captionof{figure}{MURATA reciver 1,5m}
%\label{fig:MURATA reciver 1,5m}
%\end{minipage}
%\begin{minipage}{0.5\textwidth}
%\includegraphics[width=1\textwidth%, draft
%]{Abbildungen/MessungenP1/MURATAs1,5m.png}
%\captionof{figure}{MURATA sender 1,5m}
%\label{fig:MURATA sender 1,5m}
%\end{minipage}
%\begin{minipage}{0.5\textwidth}
%\includegraphics[width=1\textwidth%, draft
%]{Abbildungen/MessungenP1/MURATAsr1,5m.png}
%\captionof{figure}{MURATA sender + reciver 1,5m}
%\label{fig:MURATA sender+reciver}
%\end{minipage}\\
%Bei den Ultraschallkapseln von EKULIT zeigte sowohl eine Verpoolung der Anschlüsse, als auch ein Tausch von Sender- und Empfängerkapsel keinen Unterschied. Bei den Ultraschallkapseln von MURATA sind bei der Wahl der Kapseln dagegen deutliche Unterschiede zu sehen. Werden nur die Sender-Kapseln als Sender und Empfänger verwendet, entsteht das in Abbildung \ref{fig:MURATA sender 1,5m} zu sehende Signal. Die Breite des eingehenden Echo Signals ist deutlich geringer, als bei den anderen verwendeten Kombinationen. Am besten sind die Signale bei der Kombination aus Sender und Empfänger von MURATA, Abbildung \ref{fig:MURATA sender+reciver} oder bei den EKULIT-Kapseln, Abbildung \ref{fig:EKULIT1,5m}.

\section{Prototyp 2}
Bei der zweiten Prototyp-Version wurden der Sender- und der Empfängerkreis auf einer Platine aufgebaut und es wurde nur noch eine Ultraschallkapsel für beide Anwendungen vorgesehen.
Um bei diesem Aufbau, einen fehlerfreien Betrieb der verwendeten voll gesteuerten Halbbrücke sicherzustellen, wurden durch den Mikrocontroller zwei getrennte PWM-Signale generiert, die wie in Abbildung \ref{fig:PWMs} zu sehen ist, durch Lücken getrennt sind. So ist sichergestellt, dass auch trotz Verzögerungen im Schaltbetrieb der Halbleiter, keine Kurzschlüsse entstehen können.\\
% Denn es darf nie außer Acht gelassen werden, dass die Einschaltzeiten von MOSFETs meistens kürzer sind, als die Ausschaltzeiten (letztere sind bis zu dreimal so lang).
\begin{minipage}{0.75\textwidth}
\includegraphics[width=1\textwidth%, draft
]{Abbildungen/MessungenP2/Zwei_PWMs_von_der_CPU.PNG}
\captionof{figure}{Verlauf der zwei generierten PWMs für den Betrieb der voll gesteuerte Halbbrücke}
\label{fig:PWMs}
\end{minipage}\\
Nachdem dieser Betrieb sichergestellt war, wurden Messungen am Verstärker (obere Linie), und am Komparator (untere Linie) vorgenommen. Dabei wurde die Verstärkung so eingestellt, dass unerwünschte Störungen gerade so nicht vom Komparator weitergegeben wurden. Die Spannung für den Sendebetrieb wurde für die Versuche zwischen 5V und 20V variiert, um betrachten zu können, wie sich das auf die Reichweite und Genauigkeit der Messungen auswirkt. Als Hindernis wurde bei allen Versuchen eine glatte Holzplatte der Maße 40x60cm verwendet und in einem Abstand von ein bis fünf Metern von der Ultraschallkapsel aufgestellt.In den Abbildungen \ref{fig:5v1m} bis \ref{fig:5v4m} sind die Ergebnisse einer Messreihe mit einer Spannung von 5V für den Sendebetrieb dargestellt. Die Ansicht wurde so eingestellt, dass zwei Sendeimpulse zu sehen sind. Dadurch wird deutlicher, welches die Sende Impulse sind, und welches die von der Entfernung abhängigen Echos sind.\\
\begin{minipage}{0.5\textwidth}
\includegraphics[width=1\textwidth%, draft
]{Abbildungen/MessungenP2/5V/1mb.PNG}
\captionof{figure}{Signalverlauf bei 5V auf 1m Abstand}
\label{fig:5v1m}
\end{minipage}
\begin{minipage}{0.5\textwidth}
\includegraphics[width=1\textwidth%, draft
]{Abbildungen/MessungenP2/5V/2mb.PNG}
\captionof{figure}{Signalverlauf bei 5V auf 2m Abstand}
\label{fig:5v2m}
\end{minipage}
\begin{minipage}{0.5\textwidth}
\includegraphics[width=1\textwidth%, draft
]{Abbildungen/MessungenP2/5V/3mb.PNG}
\captionof{figure}{Signalverlauf bei 5V auf 3m Abstand}
\label{fig:5v3m}
\end{minipage}
\begin{minipage}{0.5\textwidth}
\includegraphics[width=1\textwidth%, draft
]{Abbildungen/MessungenP2/5V/4mb.PNG}
\captionof{figure}{Signalverlauf bei 5V auf 4m Abstand}
\label{fig:5v4m}
\end{minipage}
Bei den Abbildungen ist zu sehen, dass das Echo-Signal mit zunehmender Entfernung immer schwächer wird. Bei einer Entfernung von fünf Metern (Abbildung \ref{fig:5v4m}) wird das Echo-Signal so schwach, dass die Signalstärke nach dem Komparator nicht mehr für eine eindeutige Auswertung über den Mikrokontroller ausreicht. Nachfolgend sind die Abbildungen einer Messreihe mit verschiedenen Spannungseinstellungen für den Sendebetrieb zu sehen. Anhand dieser Messreihe soll dargestellt werden, welchen Einfluss die eingestellte Spannung im Sendebetrieb auf die Reichweite des Ultraschallsignals hat. Für die Darstellung wurden die Messungen bei 5 Meter Abstand ausgewählt.\\
\begin{minipage}{0.5\textwidth}
\includegraphics[width=1\textwidth%, draft
]{Abbildungen/MessungenP2/5V/5m.PNG}
\captionof{figure}{Signalverlauf bei 5V auf 5m Abstand}
\label{fig:5v5m2}
\end{minipage}
\begin{minipage}{0.5\textwidth}
\includegraphics[width=1\textwidth%, draft
]{Abbildungen/MessungenP2/10V/5mb.PNG}
\captionof{figure}{Signalverlauf bei 10V auf 5m Abstand}
\label{fig:10v5m}
\end{minipage}
\begin{minipage}{0.5\textwidth}
\includegraphics[width=1\textwidth%, draft
]{Abbildungen/MessungenP2/15V/5mb.PNG}
\captionof{figure}{Signalverlauf bei 15V auf 5m Abstand}
\label{fig:15v5m}
\end{minipage}
\begin{minipage}{0.5\textwidth}
\includegraphics[width=1\textwidth%, draft
]{Abbildungen/MessungenP2/20V/5mb.PNG}
\captionof{figure}{Signalverlauf bei 20V auf 5m Abstand}
\label{fig:20v5m}
\end{minipage}
Bei Vergleich der Abbildungen \ref{fig:5v5m2} und \ref{fig:10v5m} mit den Abbildungen \ref{fig:15v5m} und \ref{fig:20v5m}  ist zu sehen, dassbei einem Abstand von 5 Metern erst bei einer Sendespannung von über 10V am Komparator ein über den Mikrokontroller auswertbares Signal vorhanden ist. 









%\begin{minipage}{0.5\textwidth}
%\includegraphics[width=1\textwidth%, draft
%]{Abbildungen/MessungenP2/.PNG}
%\captionof{figure}{}
%\label{fig:}
%\end{minipage}