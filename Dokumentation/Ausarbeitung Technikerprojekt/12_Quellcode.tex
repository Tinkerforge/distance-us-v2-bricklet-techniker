\begin{lstlisting}
/* distance-us-v2-bricklet*/
/* Copyright (C) 2018 Olaf Lüke <olaf@tinkerforge.com>*/

/* a16pt.c: Driver for HDC1080 humidity sensor*/
*
* This library is free software; you can redistribute it and/or
* modify it under the terms of the GNU Lesser General Public
* License as published by the Free Software Foundation; either
* version 2 of the License, or (at your option) any later version.
*
* This library is distributed in the hope that it will be useful,
* but WITHOUT ANY WARRANTY; without even the implied warranty of
* MERCHANTABILITY or FITNESS FOR A PARTICULAR PURPOSE. See the GNU
* Lesser General Public License for more details.
*
* You should have received a copy of the GNU Lesser General Public
* License along with this library; if not, write to the
* Free Software Foundation, Inc., 59 Temple Place - Suite 330,
* Boston, MA 02111-1307, USA.
*/

#include "a16pt.h"
#include <stdint.h>
#include <stdbool.h>
#include <string.h>
/*********Infineon_eigene_Include_Datein*************/
#include "xmc_gpio.h"
#include "xmc_scu.h"
#include "xmc1_ccu4_map.h"
#include "xmc_ccu4.h"
/*************Eigene_Include_Dateien****************/
#include "bricklib2/logging/logging.h"
#include "system_timer/system_timer.h"
#include "eru/eru.h"
#include "ccu4_pwm_timer/ccu4_pwm_timer.h"
#include "configs/config_a16pt.h"
int x=0;int z=0;
uint32_t v=0; uint32_t v1=0;
int zeit=0; int zeit1=0;
/*************Interrupt_FUnktionen****************/
void IRQ_Hdlr_8(void)
{
	XMC_CCU4_SLICE_StopTimer(CCU40_CC40);
}
void IRQ_Hdlr_16(void)         //TIMER Überlauf Interrupt
{
	//XMC_GPIO_ToggleOutput(P0_0);
}
void IRQ_Hdlr_3(void) //ERU
{
	/*x=0;
	XMC_CCU4_SLICE_StartTimer(CCU40_CC40); */
}
void IRQ_Hdlr_7(void) 												// Compare Interrupt
{
}
void a16pt_init(void) {
//int messwerte[10]={};
/*****************Externe_Interrupt*******************/
	eru_init(eru_port);
/************PWM_Init********************************/
	ccu4_pwm_init(pwm_port, cc40, period_);
	ccu4_pwm_set_duty_cycle(cc40, compare_);
/************Event_Config****************************/
	count_init(cc41);
	capture_init(cc43);
/*******************Timer_2_Init*******************/
	ccu4_timer_2_init(cc42);
	XMC_CCU4_SLICE_StartTimer(CCU40_CC42);
	/*
	XMC_CCU4_SLICE_StopTimer(CCU40_CC42);
	XMC_CCU4_SLICE_ClearTimer(CCU40_CC42);
	*/
/********************LED_INIT*********************/
pin_out_init(P2_0);
pin_out_init(P0_0);
XMC_GPIO_SetOutputHigh(P2_0);
XMC_GPIO_SetOutputHigh(P0_0);
/**************Taster_INIT**********************/
pin_in_pullup_init(pullup_port);
}
void a16pt_tick(void) {
	static uint32_t debug_time = 0;
	static uint32_t signal_time = 0;
	// Print every 250ms
	if(system_timer_is_time_elapsed_ms(debug_time, 250)) {
		debug_time = system_timer_get_ms();
		uint32_t slice0 = XMC_CCU4_SLICE_GetTimerValue(CCU40_CC40);
		uint32_t slice1 = XMC_CCU4_SLICE_GetTimerValue(CCU40_CC41);
		uint32_t slice2 = XMC_CCU4_SLICE_GetTimerValue(CCU40_CC42);
		uint32_t slice3 = XMC_CCU4_SLICE_GetTimerValue(CCU40_CC43);
		logd("CCU40 s0: %d, s1: %d, s2: %d, s3: %d\n\r", slice0, slice1, slice2, slice3);
	}
	if(system_timer_is_time_elapsed_ms(signal_time, 33)) {
		signal_time = system_timer_get_ms();
		XMC_CCU4_SLICE_ClearTimer(CCU40_CC41);
		XMC_CCU4_SLICE_StartTimer(CCU40_CC40);
	}
	/*
	v1=XMC_CCU4_SLICE_GetTimerValue(CCU40_CC42);
	logd("v1:%d\n\r",v1);
if(XMC_GPIO_GetInput(P2_5)==0)
void a16pt_tick(void)
{
	for(z=0;z<messwerte[z];z++)
	{
		messwerte[z]=v1;
	}
}
if(v>1)
{
	zeit=(v/46.875)*4;
	logd("zeit:%d\n\r",zeit);
	zeit1=(v1/46.875)*4;
	logd("zeit1:%d\n\r",zeit1);
	XMC_CCU4_SLICE_ClearTimer(CCU40_CC42);
	v=0;
}
*/
}
uint16_t a16pt_get_distance(void)
{
return 0;
}
\end{lstlisting}
