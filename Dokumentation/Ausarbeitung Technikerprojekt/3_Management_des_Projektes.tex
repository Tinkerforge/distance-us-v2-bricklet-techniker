
\subsection{Trello}
Zur zeitlichen Planung und Übersicht des Ablaufes wurde auf das Onlinetool Trello zurückgegriffen. Dieses ist ein kostenfreies, web basiertes Projektmanagementtool. Es ermöglicht den Gruppenmitgliedern gleichzeitig von verschiedenen Orten auf die Oberfläche zuzugreifen und Änderungen vorzunehmen. So kann ein Teilnehmer auch neue Termine mit Kennzeichnung der Fälligkeit für andere Gruppenmitglieder einfügen, oder bereits erledigte Aufgaben für alle abhaken. Auch können hier relevante Dokumente, die alle Gruppenmitglieder lesen sollen hochgeladen, und bei Bedarf noch kommentiert werden. Für die Dokumentation lässt sich an diesem System auch wunderbar abgleichen, zu welchen Zeitpunkten die einzelnen Aufgaben abgeschlossen wurden.

\subsection{Github}
Bei Github handelt es sich um einen webbasierten Online-Dienst, der Server für Software- Entwicklungsprojekte mit einer Versionsverwaltung bereitstellt. So können alle Daten nach einer Änderung im Programm wieder hochgeladen und mit einem Kommentar versehen werden. Sollte nach mehreren Änderungen ein Problem auftreten, kann einfach auf eine ältere Version zurück gegriffen und so der Fehler eingegrenzt werden. auch kann ein Projekt auf mehrere Abschnitte aufgeteilt werden, damit mehrere Personen unabhängig voneinander daran arbeiten können. Nach der Bearbeitung können die Programmteile wieder zusammengefügt werden. Dabei ist erkennbar, welche Änderungen von wem vorgenommen wurden. So können alle Vorgänge jederzeit verfolgt werden, um eine größtmögliche Übersicht zu gewährleisten. Durch das kommentieren der Änderungen kann die Nachvollziehbarkeit dieser ebenfalls deutlich gesteigert werden. Des weiteren ist diese Plattform gerade für Unternehmen wie Tinkerforge, die ihren Quellcode als Open-Source anbieten besonders reizvoll, da die Nutzer hier schnell an alle Dateien ran kommen.