
\subsection{Trello}
Zur zeitlichen Planung und Übersicht des Ablaufes wurde auf das Onlinetool Trello zurückgegriffen. Dieses ist ein kostenfreies, webbasiertes Projektmanagementtool. Es ermöglichte den Gruppenmitgliedern gleichzeitig von verschiedenen Orten auf die Oberfläche zuzugreifen und Änderungen vorzunehmen. So konnte ein Teilnehmer auch neue Termine mit Kennzeichnung der Fälligkeit für andere Gruppenmitglieder einfügen, oder bereits erledigte Aufgaben für alle abhaken. Auch konnten hier relevante Dokumente, die alle Gruppenmitglieder lesen sollten hochgeladen, und bei Bedarf noch kommentiert werden. Für die Dokumentation ließ sich an diesem System auch einfach abgleichen, zu welchen Zeitpunkten die einzelnen Aufgaben abgeschlossen wurden.

\subsection{Github}
Bei Github handelt es sich um einen webbasierten Online-Dienst, der Server für Entwicklungsprojekte mit einer Versionsverwaltung bereitstellt. So können alle Daten nach einer Änderung im Programm wieder hochgeladen und mit einem Kommentar versehen werden. Wenn nach mehreren Veränderungen ein Problem auftritt, besteht die Möglichkeit auf eine ältere Version zurückgegriffen und somit kann der Fehler eingegrenzt werden. Auch kann ein Projekt in Teilabschnitte aufgeteilt werden, damit mehrere Personen unabhängig voneinander daran arbeiten können. Nach der Bearbeitung können diese wieder zusammengefügt werden. Dabei ist erkennbar, welche Änderungen, von wem vorgenommen wurden. So können alle Vorgänge jederzeit verfolgt werden, um eine größtmögliche Übersicht zu gewährleisten. Durch das Kommentieren der Änderungen kann die Nachvollziehbarkeit dieser ebenfalls deutlich gesteigert werden. Des weiteren ist diese Plattform gerade für Unternehmen wie Tinkerforge, die ihren Quellcode als Open-Source anbieten besonders praktisch, da den Nutzern hier alle veröffentlichten Daten direkt zur Verfügung stehen.
