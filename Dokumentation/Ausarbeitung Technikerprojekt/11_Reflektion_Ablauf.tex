Bei Erhalt der Aufgabenstellung entstand bereits ein Bild der zu erledigenden Arbeiten. Dieses Bild wurde binnen kürzester Zeit deutlich umgestaltet. So war der Aufwand bei der Einarbeitung, gerade im Bereich der Programmierung deutlich höher als angenommen. Dank der tatkräftigen Unterstützung durch die betreuenden Mitarbeiter konnten größere Schwierigkeiten in diesem Bereich vermieden werden. Bezüglich der Hardwareentwicklung zeigte sich bei diesem Projekt das Problem, dass jeder Fehler im Platinenlayout ein längeres Nachspiel mit sich brachte. Erstens konnte der Fehler nach Bestellung der Platine nicht mehr korrigiert werden und die Platine musste mit Hilfe von Fädeldraht und Leiterbahnunterbrechungen angepasst werden. Zweitens betrug die Lieferzeit einer Platine immer mindestens eine halbe Woche, was heißt dass auch wenn ein Fehler bekannt war, dieser erst nach Erhalt der Platine behoben werden konnte. Dadurch ging zusätzlich Zeit verloren, die bereits für Messungen hätte genutzt werden können. Durch die kontinuierliche Mitschrift von Informationen und Stichpunkten für die Dokumentation konnte in diesem Punkt einiges an Aufwand und Zeit eingespart werden, obwohl einzelne Teile der Dokumentation komplett überarbeitet werden mussten. Auch wurden zu beginn, durch anfängliche, kleine Fehler in der Recherche falsche Bauteile für die Platine ausgewählt. Durch eine umsichtige Arbeitsweise ließ sich die Zerstörung anderer Bauteile durch Fehler größtenteils vermeiden. Durch die Aufteilung der Gruppenmitglieder, zu beginn des Projekts, auf verschiedene Arbeitsbereiche, ließen sich auch im späteren Verlauf mehrere Aufgaben parallel abarbeiten. Durch diese Aufteilung ließen sich gleichzeitig Versuche an der Testplatine und an dem Programm zur Steuerung der Platine durchführen.\\
Daraus resultierend ließ sich das Ziel des Projekts trotz Verzögerungen problemlos erreichen.