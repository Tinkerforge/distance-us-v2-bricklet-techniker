Bevor ein erster Prototyp erstellt werden konnte, mussten die richtigen Bauteile ausgewählt werden. Dies Geschah unteranderem anhand der gewonnen Informationen aus dem Kapitel 2 Vorbereitung. 

\subsection{Mikrocontroller}
Der Infineon XMC 1404\_Q048 gehört zur der Familie der ARM Cortex -M0 Prozessoren und ist ein 32-bit Industrial Microcontroller, wird mit 48~MHz externer Clock betrieben. Die 48 im Namen des Prozessors steht für die Anzahl der Pins. Der interne Timer läuft mit 96Mhz. Die CCU4 des Prozessor bietet ein 2x4 16 bit Timer. Außerdem hat der XMC einen 12 bit A/D Wandler, welcher für die Analogmessung eine genauere Auflösung bieten kann als ein 8 bit A/D Wandler. Die Betriebsspannung des Prozessors beträgt 3,3~V. Die Auswahl des Controller wurde getroffen, weil dieser standardmäßig auch schon bei Tinkerforge eingsetz wird und uns für den Prototypen vorgegeben wurde. 


\subsection{Sender}
Um eine Trennung zwischen der CPU (3,3~V) und dem Hochsetzsteller (5~V-20~V)zu ermöglichen, wurde ein weiteres Bauteil benötigt. Dafür wurde das IC A5950 (Voll Brücke) ausgewählt. Diese H-Brücke kann eine angeschlossene Last mit der vom Hochsetzstellers erzeugten Spannung versorgen. Deren Frequenz wird über das Signal an dem Anschluss Phase von der CPU vorgegeben. An den Anschlüssen Out1 und Out2 wird das Ausgangssignal abgegriffen. Für die genaue Beschaltung des ICs siehe Anhang Datenblatt “Schimatic A5950“.


\subsection{Hochsetzsteller}

Tinkerforge nutzt schon eine Variante eines Hochsetzstellers auf ihren Platinen. Der Aufbau und die Bauteilauswahl musste nur auf die variable Ausgangspannungs erweitert werden. Die Standardschaltung, von Tinkerforge, wurde an dem Eingang der Feedbackspannung mit einem Potentiometer (RV1) erweitert, um die gewünschte Variable Spannung zu generieren.\\
Die Ausgangsspannung wird mit den externen Widerständen RV1, R13 und R2 eingestellt (siehe Grundschaltung Datenblatt vom IC LRM62014x, Seite 11). Ein Wert von ca. \(\displaystyle 13,3~k\Omega \) wurde für R2 empfohlen. Mit den Potentiometer (RV1) lässt sich die Ausgangsspannung bei 5~V Eingangsspannung bis 21,5~V einstellen. In der Formel werden die Widerstände RV1 und R13 zum Ersatzwiderstand(Re) zusammengefasst.
\onehalfspacing \\
\[\displaystyle Re=R2*\left(\frac{Vout}{1,23}-1\right) \Rightarrow Vout=\left(\frac{Re}{R2}+1\right)*1,23\] 
\singlespacing
\begin{center}
\begin{minipage}{0.75\textwidth}
\includegraphics[width=1\textwidth%, draft
]{Abbildungen/Pumpe.png}
\captionof{figure}{Hochsetzsteller}
\label{fig:Hochsetzsteller}
\end{minipage}\\
\end{center}

\subsection{Ultraschallkapsel}%bearbeiten
Für den Prototypen wurden mehrere Kapseln verschiedener Hersteller bestellt. Dieses geschah um Unterschiede der verschiedenpreisigen Bauteile zu ermitteln und festzustellen welches Preissegment die nötige Qualität für die vorliegende Anwendung erfüllt.


\subsection{Filter}%bearbeiten
Um unerwünschte Signalanteile mit Frequenzen, die unter 40~kHz liegen, zu unterdrücken musste die Filterschaltung mit einem Hochpassfilter (CR Glied) bestehend aus einem Kodensator und einem Widerstand bestückt. Der Widerstand wurde nach der e24 Reihe ausgewählt.
Die Kapazität des Kondensators wurde an die Grenzfrequenz von 40~kHz und den Widerstand angepasst.
\onehalfspacing \\
\[\displaystyle C=\frac{1}{2*pi*fg*R}\Rightarrow\frac{1}{2*pi*40~kHz*100~K\Omega}\approx40~pF \]
\singlespacing
Anhand der Berechnung wurde für den Hochpassfilter ein Kondensator mit \(\displaystyle 39~pF\) genommen.

\subsection{Empfänger}
Die Abbildung \ref{fig:Empfaengerschaltung} zeigt die Empfängerschaltung. Durch diese Verschaltung von Operationsverstärkern wird das ankommende sinusförmige Signal verstärkt und in ein digitales Signal umgewandelt. 
Für die Verstärkung der Amplitude so wie der Umwandlung des analogen Signals in ein Rechtecksignal mit 40~KHz wurde der TLC272 ausgewählt, weil er günstigem Stückpreis ist und die Technischenspezifikationen ausreichend sind.
Für die Verstärkung der Amplitude ist der Operationsverstärker TLC272 U2B als nicht invertierender Verstärker geschaltet.\\

Für die Umwandlung des analogen Signales in ein digitales wurde der Operationsverstärker TLC272 U2C als Komparator geschaltet. Beim Auftreten von Differenzen zwischen den Eingangssignalen, wechselt der Ausgang des Komparators zwischen Low (0~Volt) auf High (3,3~Volt).\\ Die Referenzspannung (\(\displaystyle Uref)\)  wird durch den Spannungsteiler R9 und R8 bestimmt.
\onehalfspacing \\
\[\displaystyle Uref=\frac{Uges*R9}{R8+R9}\Rightarrow\frac{3,3~V*120~K\Omega}{100~K\Omega+120~K\Omega}=1,8~V \]
\singlespacing
\begin{center}
\begin{minipage}{0.75\textwidth}
\includegraphics[width=1\textwidth%, draft
]{Abbildungen/Empfaenger.png}
\captionof{figure}{Empfängerschaltung}
\label{fig:Empfaengerschaltung}
\end{minipage}\\
\end{center}

