\subsection{Controller}
Der Infineon XMC 1xxx48 gehört zu der Familie der ARM Cortex -M0 Prozessoren und ist ein 32-bit Industrial Microcontroller und wird mit 48MHz externer Clock betrieben. Die 48 im Namen des Prozessors steht für die Anzahl der Pins. Der interne Timer läuft mit 96Mhz. Außerdem bietet der XMC einen 12 bit A/D Wandler, welcher für die Analogmessung eine viel genauere Auflösung bieten kann als ein 8 bit A/D Wandler. Die Betriebsspannung des Prozessors beträgt 3,3V. Die Auswahl des Controller wurde getroffen, weil der standardmäßig auch schon bei Tinkerforge eingsetz wird und uns für den Prototypen vorgeben wurde.\


\subsection{Sender}
Die Auswahl für die Sender Schaltung ,mit internen MOS-Fets die Schaltzyklen mit 40kHz erlauben:eine Brückenschaltung mit internem Controller dafür eignete sich das IC A5950.
 Wie in Abbildung \ref{fig:} zu sehen ist, kann die H-Brücke eine angeschlossene Last mit einer Wechselspannung versorgen, deren Frequenz über das Signal am Anschluss PHASE vorgegeben werden kann. 
Siehe den Anhang für das Datenblatt des ICA5950 um die genauere Beschaltung des ICs

\subsection{Hochsetzsteller}
Mit einer Induktivität von \(\displaystyle 10mH\) und einem Kapazität von \(\displaystyle 40uF\) lässt sich die Ausgangsspannung bei 5V Eingangsspannung zwischen 6V und 20V einstellen.
\onehalfspacing \\
\(\displaystyle R1=R2*\left(\frac{Vout}{1,23}-1\right)\Rightarrow Vout=\left(\frac{R1}{R2}+1\right)*1,23\) 
\singlespacing


\subsection{Ultraschallkapsel}
Für den Prototypen wurden mehrere Kapseln vereschiedener Hersteller bestellt. Dieses geschah um Unterschiede der verschiedenpreisigen Bauteile zu ermitteln und festzustellen, welches Preissegment die nötige Qualität für die vorliegende Anwendung erfüllt.




\subsection{Filter}

Die Empfänger Schaltung wurde mit einem Hochpassfilter (CR Glied) bestückt bestehend aus C12 und R5 um unerwünschte Signalanteile mit Frequenzen, die unter 40kHz liegen, zu unterdrücken. Der Widerstand wurde nach der e24 Reihe ausgewählt.
Die Kapazität des Kondensators C12 wurde an die Grenzfrequenz von 40 kHz und den Widerstand angepasst.
\onehalfspacing \\
\(\displaystyle C12=\frac{1}{2*pi*fg*R}\Rightarrow\frac{1}{2*pi*40kHz*100 K\Omega}\approx40pF \)
\singlespacing
Durch nachträgliche Versuche wurde festgestellt, dass durch das erhöhen der Kapazität auch die Qualität der Filterung des Signals sich verbessert, in den folgenden Abbildungen\ref{fig:Ohne Filter}, \ref{fig:Hochpass 40pF} ,\ref{fig:Hochpass 100nF} sind die Unterschiede zusehen.

\subsection{Empfänger}
Die Abbildung \ref{fig:empfängerschaltung} zeigt die Empfängerschaltung. Durch diese Verschaltung von Operationsverstärkern(OPVs) wird das ankommende Sinusförmige Signal verstärkt und in ein digitales Signal umgewandelt. 
Für die Verstärkung der Amplitude so wie der Umwandlung des analogen Signals in ein Rechtecksignal mit 40 KHz standen zwei Operationsverstärker zur Auswahl, LT1112 mit einem Stückpreis von 4,80\euro\ und den TLC272 mit einem Stückpreis von 0,88\euro\ trotz der besseren Performanz, wurde der TLC272 für die Prototypen ausgewählt um die preislich günstigen Möglichkeiten zu prüfen. Die Versorgungsspannung der OPV's von 3,3V wird durch den Kondensator C16 (EMV Störfilter) stabilisiert.\\
Für die Verstärkung der Amplitude ist der Operationsverstärker TLC272 U2B als nicht invertierender Verstärker geschaltet.
Wenn eine Gleichspannung anliegt, wirkt der Kondensator (C10) in der Operationsverstärkerschaltung als Impedanzwandler, also mit einer Verstärkung von eins, geht nun die Eingangsfrequenz hoch, nimmt der Widerstand des Kondensators (C10) ab, somit beginnt der Operationsverstärker auch zu verstärken, und zwar mit zunehmender Verstärkung, bis irgendwann die Impedanz des Kondensators vernachlässigt werden kann und die Verstärkung nur noch durch das Verhältnis der Widerstände beeinflusst wird, R6 ist zudem notwendig um das schwingen der Amplitude zu verhindern, somit kann die Verstärkung mit folgender Formel berechnet werden:
\onehalfspacing \\
\(\displaystyle Vu= \frac{ R6+R8+R12}{R6} \) 
\singlespacing
Für die Umwandlung des Analogen Signales in ein Digitales wurde der Operationsverstärker TLC272 U2C als Komparator geschaltet. Beim Auftreten von Differenzen zwischen den eingangs Signalen, wechselt der Ausgang des Komparators zwischen Low (0 Volt) auf High (3,3 Volt).\\ Die Referenz (\(\displaystyle Uref).\) Spannung wird durch den Spannungsteiler R9 und R8 bestimmt.
\onehalfspacing \\
\(\displaystyle Uref=\frac{Uges*R9}{R8+R9}\Rightarrow\frac{3,3V*120K\Omega}{100K\Omega+120K\Omega}=1,8V \)
\singlespacing





