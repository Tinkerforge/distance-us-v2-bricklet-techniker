\subsection{Ultraschallkapsel}
Für den Sender (und empfänger) wurden auf Pietzomodulen basierende Kapsellautsprecher verwendet. Dabei wurden für den Prototypen mehrere Kapseln vereschiedener Hersteller bestellt. Dieses geschah um Unterschiede der verschiedenpreisigen Bauteile zu ermitteln und festzustellen, welches Preissegment die nötige Qualität für die vorliegende Anwendung erfüllt. Wichtig sind bei der Überprüfung das nachschwingen der Kapsel, nach Ende des zu sendenden Signals, die Empfindlichkeit auf eingehende Störfrequenzen und die zu erreichende Reichweite mit verschiedenen Signalstärken.

\subsection{Sender}
Da der Sende-, und der Empfangsbetrieb über eine Kapsel laufen sollen und die beiden Betriebsarten mit verschiedenen Spannungspegeln arbeiten, ist eine Umschaltung zwischen den Anschlüssen notwendig.\\
Zu Beginn wurde eine H-Brücke als optionale Lösung des Problems ins Auge genommen. Dafür wurde das IC A5950 verwendet. Wie in Abbildung \ref{fig:} zu sehen ist, kann die H-Brücke eine angeschlossene Last mit einer Wechselspannung versorgen, deren Frequenz über das Signal am Anschluss PHASE vorgegeben werden kann.

Später wurde auf eine Halbbrücke umgeschwenkt, bei der die MOSFETs einzeln ansteuerbar waren. Diese Halbbrücke bietet den Vorteil, dass bei der Ansteuerung der einzelnen MOSFETs gezielt Zwischenzeiten zwischen den Schaltvorgängen eingepflegt werden können, damit ein Abstand zwischen den Schaltflanken entsteht.

%Die Kapsel wird mit einer Frequenz von 40kHz und einer Amplitude von bis zu 20V P betrieben. Um ein positives sowie negatives Rechtecksignal zu generieren wird eine H-Brücke verwendet. Sie Abb.x Funktion Block Diagramm vom IC A5950\\
%In der Abb. Funktion Block Diagramm vom IC A5950 wird an OUT 1 sowie an OUT 2 eine Stromumkehrung erzielt, durch die Ansteuerung von den MOS-FET von dem Control Logic. Eine genaue Funktionsanalyse ist nicht erforderlich, weil an der H-Brücke nur ein Ultraschallsensor
%angeschlossen wird. Wichtig für die H-Brücke ist das sie mit 40kHz schalten kann die am Eingang. Ausgründen der internen Beschaltung und Toleranzen sind am Ausgang nicht die vollen 40kHz zu erwarten, sondern ein Verzug was später im Kapitel X.X behandelt wird.
\subsection{Empfänger}
Die Abbildung \ref{fig:empfängerschaltung} zeigt die Empfängerschaltung. Durch diese Verschaltung von Operationsverstärkern(OPVs) wird das ankommende Sinusförmige Signal verstärkt und in ein digitales Signal umgewandelt. Die Schaltung wurde mit einem Hochpassfilter (CR Glied) bestückt bestehend aus C12 und R5 um unerwünschte Signalanteile mit Frequenzen, die unter 40kHz liegen, zu unterdrücken. Der Widerstand wurde nach der e24 Reihe ausgewählt.
Die Kapazität des Kondensators C12 wurde an die Grenzfrequenz von 40 kHz und den Widerstand angepasst.
\onehalfspacing \\
\(\displaystyle C12=\frac{1}{2*pi*fg*R}\Rightarrow\frac{1}{2*pi*40kHz*100 K\Omega}\approx40pF \)
\singlespacing
Durch nachträgliche Versuche wurde festgestellt, dass durch das erhöhen der Kapazität auch die Qualität der Filterung des Signals sich verbessert, in den folgenden Abbildungen\ref{fig:Hochpass 40pF} ,\ref{fig:Hochpass 100nF} sind die Unterschiede zusehen. \\
Für die Verstärkung der Amplitude so wie der Umwandlung des analogen Signals in ein Rechtecksignal mit 40 KHz standen zwei Operationsverstärker zur Auswahl, LT1112 mit einem Stückpreis von 4,80\euro\  und den TLC272 mit einem Stückpreis von 0,88\euro\  trotz der besseren Performanz, wurde der TLC272 für die Prototypen ausgewählt um die preislich günstigen Möglichkeiten zu prüfen. Die Versorgungsspannung der OPV's von 3,3V wird durch den Kondensator C16 (EMV Störfilter) stabilisiert.\\
Für die Verstärkung der Amplitude ist der Operationsverstärker TLC272 U2B als  nicht invertierender Verstärker geschaltet.
Wenn eine Gleichspannung anliegt, wirkt der Kondensator (C10) in der Operationsverstärkerschaltung als Impedanzwandler, also mit einer Verstärkung von eins, geht nun die Eingangsfrequenz hoch, nimmt der Widerstand des Kondensators (C10) ab, somit beginnt der Operationsverstärker auch zu verstärken, und zwar mit zunehmender Verstärkung, bis irgendwann die Impedanz des Kondensators vernachlässigt werden kann und die Verstärkung nur noch durch das Verhältnis der Widerstände beeinflusst wird, R6 ist zudem notwendig um das schwingen der Amplitude zu verhindern, somit kann die Verstärkung mit folgender Formel berechnet werden:
\onehalfspacing \\
\(\displaystyle Vu=R6+R8+\frac{R12}{R6} .\) 
\singlespacing
Für die Umwandlung des Analogen Signales in ein Digitales wurde der Operationsverstärker TLC272 U2C als Komparator geschaltet. Beim Auftreten von Differenzen zwischen den eingangs Signalen, wechselt der Ausgang des Komparators zwischen Low (0 Volt) auf High (3,3 Volt).\\ Die Referenz Spannung wird durch den Spannungsteiler R9 und R8 bestimmt.\\\\
\(\displaystyle Uref=\frac{Uges*R9}{R8+R9}\Rightarrow\frac{3,3V*120K\Omega}{100K\Omega+120K\Omega}=1,8V \)

\subsection{Hochsetzsteller}


\(\displaystyle R1=R2*\left(\frac{Vout}{1,23}-1\right)\Rightarrow Vout=\left(\frac{R1}{R2}+1\right)*1,23\) 


\subsection{Controller}
Der Infineon XMC 1xxx48 gehört zu der Familie der ARM Cortex -M0 Prozessoren und ist ein 32-bit Industrial Microcontroller und wird mit 48MHz externer Clock betrieben. Die 48 im Namen des Prozessors steht für die Anzahl der Pins. Der interne Timer läuft mit 96Mhz. Außerdem bietet der XMC einen 12 bit A/D Wandler, welcher für die Analogmessung eine viel genauere Auflösung bieten kann als ein 8 bit A/D Wandler. Die Betriebsspannung des Prozessors beträgt 3,3V. Die Auswahl des Controller wurde getroffen, weil der standardmäßig auch schon bei Tinkerforge eingsetz wird und uns für den Prototypen vorgeben wurde.\
https://www.infineon.com/cms/en/product/microcontroller/32-bit- industrial-microcontroller- based-
on-arm- cortex-m/32- bit-xmc1000- industrial-microcontroller- arm-cortex- m0/\#\\

