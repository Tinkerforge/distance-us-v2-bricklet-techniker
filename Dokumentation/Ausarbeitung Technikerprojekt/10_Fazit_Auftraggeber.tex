Aus den Versuchen und Messungen lassen sich mehrere Aussagen treffen. \\
Als erstes, eine Ultraschall-Entfernungsmessung ist mit wenigen Bauteilen, sowohl als zwei Kapsel Variante, als auch als ein Kapsel Variante durchführbar. Bei der ein Kapsel Variante ist darauf zu achten, dass der Verstärker eine ausreichende Spannungsfestigkeit besitzt , um nicht durch das Sendersignal zerstört zu werden. Auch ist wichtig, dass bei der Erzeugung des PW-Modulierten Ausgangssignals die MOSFETs so angesteuert werden, dass das MOSFET, welches die High-Impulse schaltet, genug Zeit zum abschalten hat , bevor das MOSFET , das das LOW-Signal schaltet, einschaltet, und umgekehrt um Kurzschlüsse an dieser Stelle zu vermeiden.\\
Eine Auswertung des Echo-Signals ist prinzipiell sowohl Digital, als auch Analog möglich. Die Digitale Auswertung läuft recht simpel ab, prinzipiell muss nur die Zeit erfasst werden, die zwischen dem Senden des PWM-Signals und dem Empfangen des Echo-Signals vergeht. Die errechnete Strecke ist zu halbieren, da die vergangene Zeit sowohl den Hin-, als auch den Rückweg beinhaltet. Bei der analogen Auswertung besteht zwar die Möglichkeit über einen Frequenzvergleich auch Signale mit noch kleinerer Echo-Amplitude zu erkennen und auszuwerten, allerdings beinhaltet dieses Vorgehen einen deutlich höheren Programmieraufwandt.\\
Bei der Berechnung der Zeiten muss berücksichtigt werden, dass das eingehende Echo-Signal die Ultraschallkapsel langsam in Schwingungen versetzt, die ersten eintreffenden Schwingungen eines PWM-Signals erzeugen also kleinere Spannungssignale als die darauf folgenden. Außerdem schwingt die Ultraschallkapsel auch nach Ende des eingehenden Signals noch etwas nach, was zur Folge hat, dass das analoge Abbild des gesendeten PWM-Signals leicht versetzt und etwas verlängert wirkt. All diese Faktoren müssen für eine genauere Berechnung der Strecke, die das Signal zurück gelegt hat, berücksichtigt werden. Allein durch die Tatsache, dass das empfangene Signal an der Ultraschallkapsel derart verändert wird, ist eine absolut exakte Messung nicht möglich. Eine Beschränkung des Fehlers auf einzelne Zentimeter ist aber realisierbar.\\
%Eddy
%Die Problematik ist nun das die Zusammenschaltung von dem Empfänger mit dem Sender was zur Folge hat das ein Ausgang vom A5950 an Masse angeschlossen ist und somit ein Kurzschluss erzeugt wird was beim wechsel von der positiven zur negativen Amplitude geschieht.
%Um eine bessere Aussage zu treffen wurde eine H-Side Schaltung aufgebaut, siehe dazu das Schaltbild x.X, das IC A5950 wurde entfernt. Die H-Side konnte nur den Ultraschallsensor nicht gegen Masse schalten was zur Folge hatte das wir ein nach schwingen am Oszilloskop messen konnten siehe „Kaptiel Messung bild x.x“ 
%Um die Schwierigkeit zu lösen benötigten wir eine Halbbrücke mit einer separaten Ansteuerung ohne eine interne Logik. Somit wurde eine Halb-Brücke aufgebaut was die H-Side Schaltung ersetzt. Die Ansteuerungslogik wurde vom Prozessor XMC 1xxx48 bewerkstelligt. 
%Filterschaltung :
%Durch nachträgliche Versuche wurde festgestellt, dass durch das erhöhen der Kapazität auch die Qualität der Filterung des Signals sich verbessert, in den folgenden Abbildungen\ref{fig:Ohne Filter}, \ref{fig:Hochpass 40pF} ,\ref{fig:Hochpass 100nF} sind die Unterschiede zusehen.
%
%Anhand der Resultate wurde eine zweite Board Version erstellt mit der änderung an der Senderschaltung, Anhang Board V2.
