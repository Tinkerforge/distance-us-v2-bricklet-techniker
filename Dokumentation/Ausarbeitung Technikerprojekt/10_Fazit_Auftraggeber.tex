Aus den Versuchen und Messungen lassen sich mehrere Aussagen treffen. \\
Als erstes, eine Ultraschall-Entfernungsmessung ist mit wenigen Bauteilen, sowohl als Zwei-Kapsel-Variante, als auch als Ein-Kapsel-Variante durchführbar. Bei der Ein-Kapsel-Variante ist darauf zu achten, dass der Verstärker eine ausreichende Spannungsfestigkeit besitzt, um nicht durch das Sendersignal gestört zu werden. Es ist wichtig, dass bei der Erzeugung des Pulsweiten-Modulierten Ausgangssignals die MOSFETs so angesteuert werden, dass den MOSFETs zwischen den Schaltsignalen genug Zeit gegeben wird, um die Schaltzustände zu erreichen. Vergleiche Abbildung \ref{fig:PWMs}. Dadurch lassen sich Kurzschlüsse an dieser Stelle vermeiden.\\
Eine Auswertung des Echo-Signals ist prinzipiell sowohl digital, als auch analog möglich. Bei der digitalen Auswertung muss die Zeit erfasst werden, die zwischen dem Senden des PWM-Signals und dem Empfangen des Echo-Signals vergeht. Die errechnete Strecke ist zu halbieren, da die vergangene Zeit sowohl den Hin-, als auch den Rückweg beinhaltet. Bei der analogen Auswertung besteht zwar die Alternative über einen Frequenzvergleich auch Signale mit kleinerer Echo-Amplitude zu erkennen und auszuwerten, allerdings beinhaltet dieses Vorgehen einen deutlich höheren Programmieraufwand.\\
Bei der Berechnung der Zeiten muss berücksichtigt werden, dass das eingehende Echo-Signal die Ultraschallkapsel langsam in Schwingungen versetzt. Die ersten eintreffenden Schwingungen eines PWM-Signals erzeugen kleinere Spannungssignale als die darauf folgenden. Außerdem schwingt die Ultraschallkapsel nach Ende des eingehenden Signals noch etwas nach, was zur Folge hat, dass das analoge Abbild des gesendeten PWM-Signals leicht versetzt und etwas verlängert wirkt. All diese Faktoren müssen für eine genauere Berechnung der Strecke, die das Signal zurück gelegt hat, berücksichtigt werden. Eine exakte Entfernungsmessung ist nicht möglich, denn das empfangene Signal an der Ultraschallkapsel verändert sich. Eine Beschränkung des Fehlers auf einzelne Zentimeter ist realisierbar.\\
Mit den aktuellen, korrigierten Schaltplänen ist der Aufbau eines Ultraschall-Entfernungsmessers durchführbar. Was noch aussteht, ist eine Festlegung auf definierte Werte für die Verstärkung und die Sendespannung, um die beiden Potentiometer durch Widerstände zu ersetzen. Denn schließlich soll ein Produkt nicht erst eingestellt werden, sondern nach dem Einschalten sofort funktionsfähig sein.
