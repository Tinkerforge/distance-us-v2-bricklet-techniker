
Das Open Source ECAD-Programm KiCAD ist eine Anwendung zum Erstellen von Schaltplänen und elektronischen Leiterplatten. Mit einem solchen Programm lassen sich die erstellten Schaltpläne vor einer Leiterplattenerstellung auf Verdrahtungsfehler prüfen, um spätere Probleme zu vermeiden.

\subsection{Schaltplan}
Beim Schaltplanentwurf gilt es auf gewisse Regeln zu achten, zu dem ist auf die Übersichtlichkeit des Schaltplans zu achten.
So sollten beispielsweise Filterkondensatoren an der Spannungsversorgung des Mikrocontrollers eingeplant werden, um die Versorgungsspannung zu stabilisieren. Auch ist schon bei dem Entwurf des Schaltplans an das spätere Platinenlayout zu denken. So muss bei der Platzierung der Bauteile darauf geachtet werden, dass die Signalintegrität gewährleistet ist, und es zu keiner Potential Verschiebung kommt.\\
Im Schaltplan wurden die Baugruppen Empfänger, Sender, Hochsetzsteller und die Anschlüsse vom Controller getrennt und so positioniert, dass die beste Übersicht dargestellt wird.\\
Um die einzelnen Bestandteile separat testen zu können, wurden dem Design NULL Ohm Widerstände, an den Verbindungspunkten, hinzugefügt. Außerdem ist es dadurch möglich, eine Zerstörung einzelner Baugruppen durch einen Verdrahtungsfehler bei der Erstinitialisierung zu vermeiden.\\   

\subsection{Platinenlayout}
Beim Entwerfen eines Platinenlayouts gibt es viele Möglichkeiten ein Ergebnis zu erzielen. So können alle Bauteile so angeordnet werden, dass alle parallelen Bauteile nebeneinander aufgereiht werden und die in Reihe dazu liegenden Bauteile darunter angeordnet sind. So sähe die Platine zwar ähnlich eines Kontaktplanes aus, allerdings ist diese Variante aus Sicht der EMV nicht sonderlich empfehlenswert.\\
Eine andere Möglichkeit wäre, die Bauteile schon im Schaltplan in Gruppen zusammenzulegen und den Schaltplan auf der Platine exakt nachzubilden. Auch bei dieser Variante ergeben sich gelegentlich Probleme, was die Führung der Leitungen und vor allem den Verlauf der Ströme angeht.\\
So sollte ein Augenmerk auf den stromführenden Leitungen liegen. Je höher der Strom ist, desto breiter und kürzer ist die Leitung auszulegen, um weniger EMV-Störungen zu erzeugen. Auch sollte die Rückführung (GND) günstigerweise als eigene Leiterschicht ausgeführt werden, um einen großen Leiterquerschnitt zu gewährleisten. So kann bei der Rückführung der Ströme auch das Risiko vermieden werden, durch Bildung von größeren Schleifen, Antennen zu erzeugen. Die GND-Schicht sollte so wenig wie möglich unterbrochen werden, vor allem sind Unterbrechungen quer zur Stromflussrichtung zu vermeiden. Zusätzlich ist zu beachten, dass Kondensatoren, die der Verringerung von Störenden Spannungsschwankungen dienen, nahe schaltenden Bauteilen angebracht werden. Die optimale Platzierung ist direkt am VDD oder VCC des ICs, so dass die Leiterbahn vor dem IC mit einem höheren Querschnitt am Kondensator liegt und dann mit leicht verringertem Querschnitt an das IC angeschlossen ist.


