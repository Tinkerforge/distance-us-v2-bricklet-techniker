
Das Open Source CAD-Programm KiCAD ist eine Anwendung zum Erstellen von Schaltplänen und elektronischen Leiterplatten. Hier lassen sich auf einfache Weise Schaltpläne erstellen und nach einer Prüfung auf fehlerfreie Verdrahtung zur vereinfachten Platinenerstellung nutzen. 

\subsection{Schaltplan}
Zur Signalerzeugung wurde ein Hochsetzsteller eingesetzt, um eine höhere Spannung erzeugen zu können, als auf dem System zur Verfügung steht. So höhere Amplituden auf den Sender gegeben werden, um eine höhere Signalstärke zu erreichen. Danach wird das Signal auf eine H-Brücke geleitet, diese dient dazu aus dem Gleichsignal ein Wechselsignal mit einer Frequenz von 40kHz zu generieren. Dieses Wechselsignal wird auf den Sender gegeben um das Spannungssignal in Schallwellen umzuwandeln.\\
Wird ein zurückkommendes Signal empfangen, so wandelt der Empfänger die Schallwellen in ein Sinussignal um. Dieses Signal wird auf einen nicht invertierenden Verstärker geleitet, um die Signal-Amplitude zu verstärken und anschließend durch einen Komperator in ein Rechteckiges(digitales) Signal umgewandelt. Dieses Signal wird im Prozessor verarbeitet, um aus der Zeit zwischen senden und empfangen des Signals in die Entfernung zum betreffenden Objekt umzuwandeln.\\
Beim Schaltplanentwurf gilt es auch auf gewisse Physikalische Eigenschaften von Leitungen und Bauteilen zu achten. So sollten bei Anschluss längerer Leitungen, Condensatoren zum ausfiltern eingefangener Funksignale und anderer EMV-Belastungen, angebracht werden. Auch benötigen Bauteile wie Platinen Kondensatoren an der Spannungsversorgung, um auch kleinste Schwnakungen dieser zu vermeiden.

\subsection{Platinenlayout}
Bei dem Entwurf eines Platinenlayouts gibt es viele Möglichkeiten ein Ergebniss zu erzielen. So können alle Bauteile so angeordnet werden, dass alle parallelen Bauteile sauber nebeneinander aufgereiht werden, und die in Reihe dazu liegenden Bauteile auch darunter angeordnet sind. So sähe die Platine zwar ähnlich eines Kontaktplanes aus, allerdings ist diese Variante aus Sicht der EMV nicht sonderlich empfehlenswert.\\
Auch können die Bauteile wie im Schaltplan in Gruppen zusammen gelegt werden und der Schaltplan auf der Platine 1:1 nachgebildet werden. Auch bei dieser Variante ergeben sich gelegentlich Probleme, was die Führung der Leitungen, und vor allem den Verlauf der Ströme angeht.\\
So sollte ein Augenmerk auf den Stromführenden Leitungen liegen. Je höher der Strom ist, desto breiter ist die Leitung auszulegen und sie sollte auch möglichst kurz gehalten werden um weniger EMV störungen zu erzeugen. Auch sollte die Rückführung (GND) günstigerweise als eigene Leiterschicht ausgeführt werden, um einen großen Leiterquerschnitt zu gewährleisten. So kann bei der Rückführung der Ströme auch das Risiko vermieden werden, durch die Bildung von größeren Schleifen Antennen zu erzeugen. Die GND Schicht sollte so wenig wie möglich unterbrochen werden, vorallem sind Unterbrechungen quer zur Stromflussrichtung zu vermeiden. Zusätzlich ist zu beachten, dass Kondensatoren meißtens zur Verringerung von Störungen nahe digitalen Bauteilen angebracht werden sollten. Die optimale Platzierung ist direkt an den Pinns, so dass die Leiterbahn mit einem höheren Querschnitt auf den Kondensator geht, und dann mit leicht verringertem Querschnitt direkt auf die Pinne des IC-Bauteils verläuft.


