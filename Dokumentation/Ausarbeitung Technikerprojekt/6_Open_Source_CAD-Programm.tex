
Das Open Source CAD-Programm KiCAD ist eine Anwendung zum Erstellen von Schaltplänen und elektronischen Leiterplatten. Hier lassen sich auf einfache Weise Schaltpläne erstellen und nach einer Prüfung auf fehlerfreie Verdrahtung zur vereinfachten Platinenerstellung nutzen. 

\subsection{Schaltplan}
Beim Schaltplanentwurf gilt es auf gewisse Regeln zu achten, zu dem sollte die Übersichtlichkeit des Schaltplans nicht außer Acht gelassen werden.
So sollten Beispielsweise bei Anschluss längerer Leitungen, Kondensatoren zum ausfiltern eingefangener Funksignale und anderer EMV-Belastungen, angebracht werden. Die Bauteile sollten möglichst auch so geführt und Platziert werden, wie sie im Layout auch liegen sollten, also gemeinsame Potenzial Bezugspunkte oder Ähnliches.\\
Im Schaltplan wurden die Schlüssel Komponente Empfänger, Sender, Hochsetzsteller und die Anschlüsse an der CPU getrennt und so Positioniert, dass die Best mögliche Übersicht dargestellt wird.\\
Die Wichtigen Grenzpunkte wurden mit einem Null Ohm widerstand bestückt, um Etappen weise die Platine in Betrieb nehmen zu können. So konnten potentielle Gefahren vermieden werde.\\

\subsection{Platinenlayout}
Bei dem Entwurf eines Platinenlayouts gibt es viele Möglichkeiten ein Ergebniss zu erzielen. So können alle Bauteile so angeordnet werden, dass alle parallelen Bauteile sauber nebeneinander aufgereiht werden, und die in Reihe dazu liegenden Bauteile auch darunter angeordnet sind. So sähe die Platine zwar ähnlich eines Kontaktplanes aus, allerdings ist diese Variante aus Sicht der EMV nicht sonderlich empfehlenswert.\\
Auch können die Bauteile wie im Schaltplan in Gruppen zusammen gelegt werden und der Schaltplan auf der Platine 1:1 nachgebildet werden. Auch bei dieser Variante ergeben sich gelegentlich Probleme, was die Führung der Leitungen, und vor allem den Verlauf der Ströme angeht.\\
So sollte ein Augenmerk auf den Stromführenden Leitungen liegen. Je höher der Strom ist, desto breiter ist die Leitung auszulegen und sie sollte auch möglichst kurz gehalten werden um weniger EMV störungen zu erzeugen. Auch sollte die Rückführung (GND) günstigerweise als eigene Leiterschicht ausgeführt werden, um einen großen Leiterquerschnitt zu gewährleisten. So kann bei der Rückführung der Ströme auch das Risiko vermieden werden, durch die Bildung von größeren Schleifen Antennen zu erzeugen. Die GND Schicht sollte so wenig wie möglich unterbrochen werden, vorallem sind Unterbrechungen quer zur Stromflussrichtung zu vermeiden. Zusätzlich ist zu beachten, dass Kondensatoren meißtens zur Verringerung von Störungen nahe digitalen Bauteilen angebracht werden sollten. Die optimale Platzierung ist direkt an den Pinns, so dass die Leiterbahn mit einem höheren Querschnitt auf den Kondensator geht, und dann mit leicht verringertem Querschnitt direkt auf die Pinne des IC-Bauteils verläuft.


