\tikzstyle{block} = [draw, fill=blue!20, rectangle, 
    minimum height=3em, minimum width=6em]
\tikzstyle{pinstyle} = [pin edge={to-,think,black}]
\tikzstyle{input} = [coordinate]
\tikzstyle{output} = [coordinate]
\tikzstyle{line} = [ draw, dashed, line width = 0.5pt, -latex']

\begin{tikzpicture}[auto, node distance=2cm,>=latex']
\node [block](controller){Controller};
\node [block, right of=controller, node distance=4.5cm] (halb-bruecke)  {Halb-Bruecke};
\node [block, above of=halb-bruecke, node distance=2cm] (hochsetzsteller) {Hochsetzsteller};
\node [block, right of=halb-bruecke, node distance=4.5cm] (ultraschallkapsel)  {Ultraschallkapsel};
\node [block, below of=ultraschallkapsel, node distance=2.5cm] (filterung)  {Filterung};
\node [block, left of=filterung, node distance=4.5cm] (verstaerker)  {Verstaerker};
\node [block, left of=verstaerker, node distance=4.5cm] (komperrator)  {Komperrator};

\draw [->] (controller) -- node[name=PWM ] {$PWM$}  (halb-bruecke);
\node [output, right of=halb-bruecke] (output) {};
\coordinate [below of=PWM] (tmp);
\draw [->] (hochsetzsteller) -- node[name=VBB ] {$VBB$}  (halb-bruecke);
\node [output, above of=halb-bruecke] (output) {};
\coordinate [below of=VBB] (tmp);
\draw [->] (halb-bruecke) --  (ultraschallkapsel);
\draw [->] (ultraschallkapsel) --  (filterung);
\draw [->] (filterung) --  (verstaerker);
\draw [->] (verstaerker) --  (komperrator);
\node [output, right of=halb-bruecke] (output) {};
\coordinate [below of=PWM] (tmp);
\draw [->] (komperrator) --  node[name=Digital ] {$Digital$} (controller);
\node [output, above of=komperrator] (output) {};
\coordinate [below of=Digital] (tmp);
\path[line] (verstaerker)--node[name=analog] {$Analog$}  (controller);

%\draw [->] (verstaerker) --  node[name=Analog ] {$Analog$} (controller);
%\node [output, above of=verstaerker] (output) {};
%\coordinate [below of=Analog] (tmp);
\end{tikzpicture}

\subsection{Sender}
Der Sender soll Schallwellen im Ultraschallbereich aussenden, die einen ausreichenden Schalldruck haben, um von Hindernissen, die bis zu mehrere Meter entfernt sind ein Echo erhalten zu können.
Dafür muss die Lautsprecherkapsel mit einem Sinusähnlichen Signal angesteuert werden, dessen Amplitude angemessen hoch ist, um den gewünschten Schalldruck zu erzeugen. Mit dem Mikrocontroller kann ein Rechtecksignal erzeugt werden, bei dem die Pulsweite variierbar ist (PWM). Der Nachteil ist, dass der Mikrocontroller nur auf seine Spannungsebene von 3,3V begrenzt ist und durch höhere Spannungen zerstört würde.
Um höhere Spannungen an der Lautsprecherkapsel realisieren zu können, muss ein weiteres Bauteil dazwischen geschaltet werden. Dieses zusätzliche Bauteil dient als Trennung zwischen den 3,3V des Mikrocontrollers und der höheren Spannungsebene der Lautsprecherkapsel. Dabei wird darauf zu achten sein, dass dieser Schalter schnell genug arbeitet, um die 40kHz (Frequenz von Ultraschall) auch sauber schalten zu können.

\subsection{Empfänger}
Im Empfangsbetrieb werden die zurückkommenden Schallsignale, die auf die Lautsprecherkapsel treffen in Sinusförmige Spannungssignale umgewandelt. Die Amplitude dieser Signale hängt von dem Schalldruck der empfangenen Signale ab und ist deutlich nieriger als die Amplitude der gesendeten Signale. Diese Signale müssen anschließend verstärkt werden, um sie mit dem Mikrocontroller auswerten zu können. Zur Vereinfachung der Auswertung macht es auch Sinn, das eingehende analoge Signal in ein Digitales Signal umzuwandeln, dies kann einiges an Verarbeitungsaufwand einsparen. Zusätzlich ist das eingehende Signal zu filtern, da die Lautsprecherkapsel nicht ausschließlich Ultraschallsignale aufnimmt.
Mit der Zeit, die zwischen dem gesendeten Signal und dem empfangenen Signal vergeht, muss letztendlich der Abstand zwischen dem Sensor und dem Hinderniss berechnet werden.
